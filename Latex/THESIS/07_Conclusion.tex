%7. Conclusions: Summarise the main findings of your project and what you have learnt. Highlight your achievements, and note the primary limitations of your work. You can also describe avenues for future work.
\section{Conclusion}\label{Conclusion}
By analysing the appearance of agile related keywords on institutional web sites over time, this study explored the evolution and spread of agile methods in the British and German government. While the two countries' institutions share some common features like an overall increasing publication trend of agile sites and a majority of these sites also mentioning terms connected to digitalisation, the numbers clearly identify the Britons as the leading agile innovator: Not only have they started publishing agile related content many years before their German peers, the spread of the methodology amongst the studied government institution is also wider, and the respective method descriptions are considerably more detailed. Although it turned out that the British government's digital service had played the leading role, the Cabinet Office's promotion of agile methods also proved to be very substantial. In order to catch up, German government could apply some of the leader's strategies like giving the methodology more weight in strategic papers on digital transformation or establishing agile lighthouse institutions.