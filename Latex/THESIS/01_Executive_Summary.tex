%1. Abstract: should be a concise (less than 300 words) motivation of the problem, description of the aims and your contribution, and main findings.
% in the Guidelines this it says this must be called ececutive summary (1 pg.) 


\section*{Executive Summary}

%"an executive summary, unlike an abstract, is a document in miniature that may be read in place of the longer document

This thesis project developed an online data collection and analysis system to trace the evolution and spread of agile methods in British and German government institutions. Agile methods are rooted in the software development departments of the private sector and are characterized by an iterative development process that focuses on the user. As the digital transformation gains ground in the public sector, these methods are also becoming more relevant to government institutions. Yet, public management literature still lacks an understanding of the temporal and spatial evolution of this development. To fill this research gap, this study analysed the occurrence of agile methods related keywords on British and German government websites over time. A total of 49 government domains were crawled and 171,569 potentially agile related pages downloaded. After preprocessing (e.g. extracting the publishing organisation's name and the publishing date) and cleaning the data, 451 relevant sites were left. The analysis showed that the number of agile related sites published by government institutions as well as the number of government domains publishing agile sites increased over time and particularly in recent years (289\% increase in number of published sites from 2017 to 2019). Furthermore, it revealed that 84\% of agile sites also mention digital transformation related keywords. German federal and state level institutions published a total of 74 agile related pages, the first one in 2015, and the most active bodies turned out to be the Federal Ministry of Labour \& Social Affairs and the state of Baden-Wuerttemberg. The Federal Government Office and the Bavarian government lead in terms of most detailed methods description. On the other hand, Britons have been discussing agile methods on their government web presences since 2011 (380 pages in total), while the lead institutions in terms of number of sites published are the Government Digital Service and the Cabinet Office. With regard to the depth of agile methods descriptions the leaders are the Home Office and the Education \& Skills Funding Agency. When comparing British and German central government ministerial departments, the wide lead – in terms of all three, quantity, timing, and depth – of the British institutions becomes apparent. To catch up, Germans should more strongly acknowledge the strategic relevance of agile methods for government's digital transformation, and consider establishing distinct lighthouse institutions that internally push the application of agile methods and openly talk about it so that other institutions can follow. Future work could include further relevant actors such as NGOs, consultancies or media outlets in the analysis to possibly reveal who "drives" the adoption of agile methods in government institutions.