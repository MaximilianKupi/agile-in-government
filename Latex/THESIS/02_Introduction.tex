%2. Introduction: Explains the problem; why it’s difficult, interesting, or important; how and why current methods fail/succeed at the problem, and explains key ideas of your approach and results. Roughly the same material as the abstract but with more space for motivation, detail, and references to existing work, and aims to capture the reader’s interest.

\setlength{\epigraphwidth}{0.5\textwidth}
\epigraph{We will always be looking to improve, because the expectations of citizens and the demands on services will always be shifting.}{Kevin Cunnington, Director General, Government Digital Service, UK (\cite*{Cunnington2017})}

\section{Introduction}
\label{Introduction}
The digital transformation is continuously changing more and more aspects of contemporary life. From the way people communicate to the way they find the right path or even the love of their life – nothing remains untouched by digital technology. In line with this digital interaction channels have seen an increase in relevance for the private and public sector alike \parencite{Koeze2020, WorldEconomicForum2020}.

In order to digitally transform any organisation, a simple transfer of analog practices into the digital realm will not do: Much rather digital transformation requires organisational and relational changes in order to better satisfy the users' needs and fully reap the benefits that digital technologies can bring \parencite{Mergel2019a}. In that respect, the adoption of agile methods has increasingly become the mantra for many public sector institutions \parencite{Lundqvist2016, Vacari2015, Mergel}.

Agile approaches originate from the IT departments of private sector companies, and are characterised by the incremental and iterative development, testing, and improvement of technology products with a focus on the user \parencite{Mergel2016}. However, as more and more non-IT related teams came to value digital technologies as integral part of their work, agile methods such as Scrum or Kanban began to expand beyond these departments. In that respect various digital government units such as United Kingdom's, United States' or Canada's government digital service are regarded as the lead organisations that "paved the way" for other government departments to also introduce agile methods (\cite[2]{Mergel}, see also \cite{Clarke2019}). However, apart from these prominent examples, canonical public administration literature knows little about the extent to which these methods are actually prevalent in other government institutions \parencite{CarvalhoFernandes2016, Vacari2015, Mergel2018, Mergel}. To help close this research gap, this study will analyse the appearance of agile related keywords on British and German government websites over time and answer the following research question:

\begin{addmargin}[2em]{2em}% 1em left, 2em right
\textbf{RQ:}\textit{What is the evolution and spread of agile methods in British and German government institutions?\footnote{When talking about the United Kingdom or Britain, this paper refers to all the government institutions that have their web presences on \href{https://gov.uk}{gov.uk}, although, strictly speaking, \href{https://gov.scot}{gov.scot} (Scotland), \href{https://gov.wales}{gov.wales} (Wales), and  \href{https://gov.ie}{gov.ie} (Ireland) are also all part of the United Kingdom.}}\label{RQ1}
\end{addmargin}\par 

With its much cited digital transformation efforts, the United Kingdom will thereby serve as the baseline against which German government institutions will be assessed \parencite{Sivarajah2014, Clarke2019}. To provide the most legitimate basis for comparison only federal government ministries' and the cabinet / federal government offices' web sites will be analysed for the inter-country comparison. However, in order to gain further insights, subordinate British institutions and German federal state ministries will also be included for the respective intra-country comparisons.

The remainder of this paper is structured as follows: The \hyperref[Literature Review]{literature review} provides an introduction to public sector's \hyperref[Digital Transformation]{digital transformation} as well as to the respective British and German policy setting. Furthermore, the characteristics of \hyperref[Agile Methods]{agile methods} as well as the connected \hyperref[Benefits, Challenges, and Conditions]{implications for their application} are discussed, and the \hyperref[Research Gap]{research gap} is specified. The 
\hyperref[Study Design]{study design} as well as the \hyperref[Data Collection]{data collection}, \hyperref[Data Preprocessing]{preprocessing}, and \hyperref[Data Analysis and Visualisation]{analysis} methods to answer the research question are described in \hyperref[Methods]{Section 3}. Additionally, this \hyperref[Methods]{section} outlines the \hyperref[Prototyping and Testing]{prototyping and testing} work that went into the development of this analysis pipeline. \hyperref[Analysis]{Section 4} presents the analysis results on a \hyperref[General Level]{general level} as well as for the \hyperref[Intra-Country Comparison]{intra-} and \hyperref[Inter-Country Comparison]{inter-country comparisons}. The \hyperref[Discussion]{discussion section} \hyperref[Summary of Findings]{summarizes the findings}, derives \hyperref[Policy Implications]{policy implications}, and points out alleys for \hyperref[Limitations and Future Work]{future research based on this studies limitations} and beyond. The paper ends with a short \hyperref[Conclusion]{conclusion}.



%FOR PhD: To answer these questions we apply a theory of gradual institutional change. (Greve2019)