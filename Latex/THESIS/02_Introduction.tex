%2. Introduction: Explains the problem; why it’s difficult, interesting, or important; how and why current methods fail/succeed at the problem, and explains key ideas of your approach and results. Roughly the same material as the abstract but with more space for motivation, detail, and references to existing work, and aims to capture the reader’s interest.


%Mergel Defining Digital Transformation: Digital transformation approaches outside the public sector are changing citizens' expectations of public administrations' need to de- liver high-value, real-time digital services. Triggered by supranational agreements, such as the “Tallinn Declaration on eGovernment” (European Commission, 2017), governments are changing their mode of operation in order to improve service delivery, be more efficient and effective in their designs, and achieve objectives such as increased transparency, interoperability, and citizen satisfaction.



%Mergel Digital Transformation: %We believe that additional research is necessary to distinguish digital transformation approaches in practice as they relate to their digital agendas. This well help identify how digital transformation differs based on the size of the country, its history, and present context as well as how these dimensions might have an impact on their digital trans- formation efforts.
%It might also be useful to break down the definitions by type of public sector services and its subsectors: We suggest that there are likely differences across sectors, such as health, traffic, safety, or social ser- vices. There might be sectors that are more prone to engage in the use of new technologies based on the types of public servants they hire. For example, national security might hire more engineers with a different type of education in technology than social services. We might see a more nuanced definition of digital transformation as a result of this distinction and the heterogeneous nature of the public sector itself.


\section{Introduction}

%why Britain and Germany?
%what is my research question

%RQ1: What is the evolution and spread of agile governance methods and principles in the German and British public administration?