\section[Literature Review]{Literature Review\footnote{Parts of this section have been previously submitted for assessment in a paper on "Applying Agile Development Methods in E-Government Projects" for Professor Dr. Mehmet Akif Demircioglu at the Lee Kuan Yew School of Public Policy, Singapore, as well as in the author's application for a PhD at the Centre for Digital Governance of the Hertie School, Berlin.}}
\subsection{Digital Transformation}\label{Digital Transformation}
The term digital transformation has become ever more prevalent in the private and lately also in the public sector to describe the ongoing changes enabled by the adoption of digital technologies (see for example \cite{McKinsey2018,BehordenSpiegel2020,Tabrizi2019}). Based on interviews with 40 experts from the field, Mergel, Edelmann, and Hauga (\cite*[p. 12]{Mergel2019a}) define the digital transformation of public administrations as "a \textit{holistic} effort to revise core processes and services of government beyond the traditional digitization efforts" [emphasis added]. To further specify the term, they differentiate digital transformation from the two connected concepts of digitization and digitalization (\cite{Mergel2019a}; see also \cite{Bloomberg2018, Brennen2015}).\par 

While digitization depicts the one-to-one transition of analogue services into the digital realm, digitalization goes further as it focuses "on potential changes in the processes beyond mere digitizing of existing processes and forms" (\cite[p. 12]{Mergel2019a}). Digital transformation in contrast, encompasses both these elements but puts a strong emphasis on organizational and relational changes of the digitally transforming entity in order to better satisfy the user needs (\cite{Mergel2019a}; see also \cite{Vial2019, Henriette2015}). In line with that, Mergel, Edelmann, and Hauga (\cite*{Mergel2019a}) identify the change in culture, skills, and mindset as an essential condition to make the digital transformation of public administrations last (see also \cite{Dierks2020, Parcell2013}). With their strong focus on staying responsive to change and satisfying user needs efficient- and effectively, adopting agile methods (see Section \ref{Agile Methods}) is considered key (\cite{Mergel2018,Beyer2010, Mergel2019, osmundsen2020, Andriole2018}).\par 
At the European level the "Tallinn Declaration on eGovernment" (\cite*{EuropeanCommission2017}) "marks a new political commitment [...] towards ensuring high quality, user-centric digital public services for citizens [...]" (\cite{EuropeanCommission2017a}). Recognizing the need to "increase the digital leadership skills" and "modernise the design of public services, procurement and contracting arrangements, to make them compatible with modern and agile ways of developing and deploying digital technology" on the country-level (\cite[p. 7]{EuropeanCommission2017}), the declaration can be seen as the latest supranational push towards the digital transformation of public administration (\cite{Mergel2019a}). Additionally, the European Union (\cite*{EuropeanUnion2018}) passed the Single Digital Gateway Regulation in order to "facilitate online access to the information, administrative procedures and assistance services" for citizens and businesses in all EU member states (\cite{EuropeanCommission2018a}).\par 
In line with these mandates, Germany released the Act for the Improvement of Online Access to Administration Services (Onlinezugangsgesetz), which requires all public services to be available digitally until 2022 (\cite{DeutscherBundestag2017}). Addressing nearly 600 public services to be digitalized in a user-centred manner, this law can be considered the main driver for public administration's digital transformation in Germany up until today (\cite{Mergel2019, BundesministeriumdesInnerenfurBauundHeimat2017}; see also \cite{EuropeanCommission2019a}).\par 
Yet, compared to the United Kingdom, Germany's current efforts still are more focused on technology and digitalization alone, and put less weight on truly transforming the public administration (\cite{Mergel2019, EuropeanCommission2019b}). Already bearing "transformation" in its title, the vision of UK's Government Transformation Strategy is to "transform the relationship between citizens and the state - putting more power in the hands of citizens and being more responsive to their needs" (\cite{CabinetOffice2017}). In order to achieve this, the strategy puts a strong emphasis on growing "the right people, skills and culture", and clearly acknowledges the importance of agile methods (\cite{CabinetOffice2017}).\par 


%-------------------------------------------------
\subsection{Agile Methods}\label{Agile Methods}
Traditionally, the development of IT related projects in public administrations and elsewhere followed a waterfall approach: Each of the phases in the development process (see \hyperref[fig:Waterfall development]{Figure 1}) is tackled sequentially, one at a time (\cite{Kannan2014, Sherrell2013}). 
%
\begin{figure}[ht!]
	\centering
	\includegraphics[height=0.25\textwidth]{{"Latex/THESIS/Figures/Waterfall"}.png}
	\caption[Waterfall development process]{Waterfall development process (adapted from \cite{Mergel2016})}
	\label{fig:Waterfall development}
\end{figure}
%
The core belief behind this approach is that by eliminating any possible mistakes within each phase, the subsequent phases "won't be impacted by mistakes and the project team won't lose time and money by going back to fix the mistakes" (\cite[p. 517]{Mergel2016}). As a consequence of this rigid structure, the waterfall process is non-responsive to sudden changes in needs and requirements (\cite{Kannan2014}). Furthermore, evaluations of the project status through user testing of the product or service only happen close to the final launch in the later stages of the process. It is only then that the project team might come to realize major errors, which in some cases consequently leads to a complete failure as in the case of the US' \href{www.healthcare.gov}{HealthCare.gov} project (\cite{Mergel2016}). All in all, these properties make the waterfall approach rather unsuited for large projects, whose nature rarely turns out to be truly sequential in reality (\cite{Kannan2014}).
 \par 
Agile development methods, in contrast, involve creating, testing, and improving technology products incrementally (\cite{Mergel2016}). The first instance of an agile method goes back to physicist and statistician Walter Shewhart of the Bell Labs, who developed the Plan Do Study Act (PDSA) approach already back in the 1920s to iteratively improve products and processes by responding to changes and new findings throughout the whole development process (\cite{Goldman1994}). Examples of currently used agile methods include Scrum, Kanban, eXtreme programming (XP), Lean Software Development, and Feature-Driven Development (FDD) (\cite{Dingsoyr2012, Rigby2016a}). Design Thinking is often seen as a useful complement to further foster the user-centredness of the solution, although, strictly speaking, the method goes beyond the core of agile development methods (\cite{DaSilva2011, AustralianDigitalTransformationAgency2019}).\par  
By and large, all agile methods adhere to the tenets of the Agile Manifesto (\cite*{AgileManifesto2001}), which was formulated by 17 thought-leaders in 2001 (\cite{Rigby2016a}). The core values of this manifesto are depicted in \hyperref[tab: Values of agile software development]{Table 1} (\cite{Dingsoyr2012}). The first value emphasises the importance of the people in the development team and their continuous interaction and communication. Valuing this over a pedantic obsession with processes and tools makes the team more likely to respond to changes quickly. Value number two highlights the adoption of a "lean" mentality when it comes to minimizing unnecessary work, in particular with regards to wasteful documentation procedures. This does not mean eliminating documentation completely, but streamlining it in a way that provides the developer what is needed to do the work without getting bogged down in minutiae (\cite{Eby2016}). The third value stresses the importance of collaboration with users and stakeholders throughout the whole development process. This way it is ensured, that their needs and concerns are being addressed by the development team. Finally, the fourth value acknowledges that uncertainties are a constitutive part of software development, and the team should stay flexible throughout the process to make changes.\par
%
\begin{table}[ht!]
	\centering
	\caption{Values of agile software development}\label{tab: Values of agile software development}
	\renewcommand{\arraystretch}{1.4}
	\begin{tabular}{ p{1.5cm} p{9.5cm} }
		\hline
		Value 1 & Individuals and interactions over processes and tools\\
		Value 2 & Working software over comprehensive documentation\\
		Value 3 & Customer collaboration over contract negotiation\\
		Value 4 & Responding to change over following a plan\\
		\hline
		\multicolumn{2}{r}{Source: \cite{AgileManifesto2001}.}	
	\end{tabular}
\end{table}
%
To achieve this level of responsiveness to change, the agile development process is divided into so called "sprint cycles" (see \hyperref[fig:Agile development process]{Figure 2}). At the beginning of the development process the project team conducts qualitative field research – potentially using Design Thinking methods – to gather the needs of the users and document them in so called "user stories" in the language of the users to avoid misconception and translation errors (\cite{Wirdemann2017}). In each of the iterations or "sprints" the project team develops increments of working software that solve these needs of the user and are the pieces of the puzzle that ultimately make up the final service or product. Each of these sprints usually takes between one to eight weeks and follows the phases of the traditional waterfall approach, however, with a strong focus on iteratively testing the increment (\cite{Hughes2013}). Small prototypes are used to rapidly prove (or disprove) ideas before much is invested, following the motto of failing fast, early and often (\cite{Boehmer2017}). The initial project plan serves as a guidance. However, the early evaluations reveal which requirements might not have been fully addressed and need to be added as work packages to the next sprint cycles, leading to a revision of the plan after each iteration (\cite{Mergel2016}).\par 
%
\begin{figure}[ht!]
	\centering
	\includegraphics[width=0.6\textwidth]{{"Latex/THESIS/Figures/Agile"}.png}
	\caption[Agile development process]{Agile development process (Author's visualisation)}
	\label{fig:Agile development process}
\end{figure}
%
Throughout the whole process, the development team is self-organizing in an environment that enhances creativity and spontaneity. To foster self-organizing teams similar to the ones in start-ups, three conditions have to be met (\cite{TakeuchiI1986}): First, \textit{autonomy} as to how exactly they plan to meet the user needs and requirements with the service or product. Second, \textit{self-transcendence} by establishing their own goals and elevating them continuously based on user feedback as the service or product takes shape. Third, \textit{cross-fertilization} achieved through an interdisciplinary, cross-agency team constellation that unites members with varying functional specializations, thought process, behaviour patterns and skill-sets. To enhance creativity and spontaneity, the management has to exercise "subtle control". This means establishing enough checkpoints to prevent ambiguity, and tension from turning into chaos, while at the same time avoiding the kind of rigid control that impairs the flow of the creative juices (\cite{TakeuchiI1986}). Typically, the team's self-organization is institutionalized by regular project meetings, as for example the daily stand-ups or the sprint planning and review meetings at the beginning and end of each sprint cycle in the case of the Scrum methodology (\cite{Scrum.org2019}). The 12 principles of the Agile Manifesto (\cite*{AgileManifesto2001}) depicted in Table~\ref{tab: Principles of agile software development} summarize the aforementioned particularities of agile methods in a comprehensive manner.\footnote{The wording of the principles (e.g. "business", "customer") suggests the methodology's origin in the private sector domain. Nevertheless, to not introduce any distortions, the original words have been preserved. Furthermore, the concrete "translation" into public sector terms depends on the specific context of each project, and thus would not have provided substantive value for the purpose of this paper.}

\begingroup
\begin{spacing}{.75}
\renewcommand
\arraystretch{2}
\begin{longtable}[ht!]{p{0.14\textwidth} p{0.79\textwidth}}
\caption{Principles of agile software development}\label{tab: Principles of agile software development}
\renewcommand{\arraystretch}{1.4}\\
\hline
Principle 1	& Our highest priority is to satisfy the customer through early and continuous delivery of valuable software.\\
Principle 2	& Welcome changing requirements, even late in development. Agile processes harness change for the customer's competitive advantage.\\
Principle 3 &	Deliver working software frequently, from a
couple of weeks to a couple of months, with a preference to the shorter timescale.\\
Principle 4	& Business people and developers must work
together daily throughout the project.\\
Principle 5	& Build projects around motivated individuals.
Give them the environment and support they need, and trust them to get the job done.
\\
Principle 6	& The most efficient and effective method of
conveying information to and within a development team is face-to-face conversation.
\\
Principle 7	& Working software is the primary measure of progress.\\
Principle 8	& Agile processes promote sustainable development. The sponsors, developers, and users should be able to maintain a constant pace indefinitely.
\\
Principle 9	& Continuous attention to technical excellence
and good design enhances agility.\\
Principle 10 & Simplicity – the art of maximizing the amount of work not done – is essential.\\
Principle 11 & The best architectures, requirements, and designs emerge from self-organizing teams.
\\
Principle 12 & At regular intervals, the team reflects on how to become more effective, then tunes and adjusts its behaviour accordingly.\\
\hline
\multicolumn{2}{r}{Source: \cite{AgileManifesto2001}.}	
\end{longtable}
\end{spacing}
\endgroup

\subsection{Related Research}
Due to its growing relevance for practitioners, the application of agile methods in the public sector has also increasingly become the focus of researchers around the world (\cite{CarvalhoFernandes2016, Mergel2018, Vacari2015}). Recent literature reviews suggest that most of the studies employ case studies, yet some also make use of surveys or interviews to derive their insights (\cite{Vacari2015, Mergel2018}). Mergel, Gong, and Bertot (\cite*{Mergel2018}) find that research focuses on four application areas of agility in government: (1) agile software development, (2) agile project management, (3) agile acquisition, and (4) agile evaluation. While agile software development (1) is the most researched area, studies that evaluate agile approaches (4) are particularly sparse; area two and three are decreasing in relevance respectively (\cite{Mergel2018}). Yet again, studies which focus on agile software development can be categorized into three main groups according to their respective focus: reasons and benefits, problems and challenges, and lessons learned obtained from adopting agile methods in the public sector (\cite{Vacari2015}). While all these studies uncover useful insights on the micro / organizational level, macro level studies that quantitatively analyse the evolution and spread of agile methods on an intra- and inter- country level for Germany and the UK are still lacking (\cite{Vacari2015, Mergel2018}). This paper aims to close this research gap.