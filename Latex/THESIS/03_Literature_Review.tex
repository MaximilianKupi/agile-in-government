\section{Literature Review}\label{Literature Review}
\subsection{Digital Transformation}\label{Digital Transformation}
The term digital transformation has become ever more prevalent in the private and lately also in the public sector to describe the ongoing changes enabled by the adoption of digital technologies (see for example \cite{McKinsey2018,BehordenSpiegel2020,Tabrizi2019}). Based on interviews with 40 experts from the field, Mergel, Edelmann, and Hauga (\cite*[p. 12]{Mergel2019a}) define the digital transformation of public administrations as "a \textit{holistic} effort to revise core processes and services of government beyond the traditional digitisation efforts" [emphasis added]. To further specify the term, they differentiate digital transformation from the two connected concepts of digitisation and digitalisation (\cite{Mergel2019a}; see also \cite{Bloomberg2018, Brennen2015}).\par 

While digitisation depicts the one-to-one transition of analogue services into the digital realm, digitalisation goes further as it focuses "on potential changes in the processes beyond mere digitising of existing processes and forms" (\cite[p. 12]{Mergel2019a}). Digital transformation, in contrast, encompasses both these elements, but puts a strong emphasis on organisational and relational changes of the digitally transforming entity in order to better satisfy the user needs (\cite{Mergel2019a}; see also \cite{Vial2019, Henriette2015}). In line with that, Mergel, Edelmann, and Hauga (\cite*{Mergel2019a}) identify the change in culture, skills, and mindset as an essential condition to make the digital transformation of public administrations last (see also \cite{Dierks2020, Parcell2013}). With their strong focus on staying responsive to change and satisfying user needs efficient- and effectively, adopting agile methods (see Section \ref{Agile Methods}) is considered key (\cite{Mergel2018,Beyer2010, Mergel2019, osmundsen2020, Andriole2018}).\par 
At the European level the "Tallinn Declaration on eGovernment" (\cite*{EuropeanCommission2017}) "marks a new political commitment [...] towards ensuring high quality, user-centric digital public services for citizens [...]" (\cite{EuropeanCommission2017a}). Recognising the need to "increase the digital leadership skills" and "modernise the design of public services, procurement and contracting arrangements, to make them compatible with modern and agile ways of developing and deploying digital technology" on the country-level (\cite[p. 7]{EuropeanCommission2017}), the declaration can be seen as the latest supranational push towards the digital transformation of public administration (\cite{Mergel2019a}). Additionally, the European Union passed the Single Digital Gateway Regulation in order "to provide access to information, to procedures and to assistance and problem-solving services" for citizens and businesses in all EU member states (\cite{EuropeanUnion2018}).\par 
In line with these mandates, Germany released the Act for the Improvement of Online Access to Administration Services (Onlinezugangsgesetz), which requires all public services to be available digitally until 2022 (\cite{DeutscherBundestag2017}). Addressing nearly 600 public services to be digitalised in a user-centred manner, this law can be considered the main driver for public administration's digital transformation in Germany up until now (\cite{Mergel2019, BundesministeriumdesInnerenfurBauundHeimat2017}; see also \cite{EuropeanCommission2019a}).\par 
However, compared to the UK, current efforts in Germany focus mostly on technology and place less emphasis on a real digital transformation of the public administration (\cite{Mergel2019, EuropeanCommission2019b}). Already bearing "transformation" in its title, the vision of UK's Government Transformation Strategy is to "transform the relationship between citizens and the state – putting more power in the hands of citizens and being more responsive to their needs" (\cite{CabinetOffice2017}). In order to achieve this, the strategy puts a strong emphasis on growing "the right people, skills and culture", and clearly acknowledges the importance of agile methods (\cite{CabinetOffice2017}).\par 


%-------------------------------------------------
\subsection[Agile Methods]{Agile Methods\footnote{This section has been adapted from a passage the author previously submitted for assessment in a paper on "Applying Agile Development Methods in E-Government Projects" for Professor Dr. Mehmet Akif Demircioglu at the Lee Kuan Yew School of Public Policy, Singapore.}}\label{Agile Methods}
\begin{wrapfigure}[8]{r}{0.5\textwidth}
	\centering
	\includegraphics[height=0.25\textwidth]{{"Latex/THESIS/Figures/Waterfall"}.pdf}
	\caption[Waterfall development process]{Waterfall development process (adapted from \cite{Mergel2016})}
	\label{fig:Waterfall development}
\end{wrapfigure}
Traditionally, the development of IT related projects followed the linear waterfall approach: Each phase in the development process (see \hyperref[fig:Waterfall development]{Figure 1}) is tackled sequentially, one at a time (\cite{Kannan2014, Sherrell2013}). The core assumption behind this approach is that by eliminating any possible mistakes within each phase, the subsequent phases "won't be impacted by mistakes and the project team won't lose time and money by going back to fix the mistakes" (\cite[p. 517]{Mergel2016}). As a consequence of this rigid structure, the waterfall process is unresponsive to sudden changes in needs and requirements (\cite{Kannan2014}). Moreover, evaluations of the project status through user testing of the product or service only happen close to the final launch in the later stages of the process. Only then, the project team might come to realize major flaws, which in some cases consequently leads to a complete failure as in the case of the US' \href{www.healthcare.gov}{HealthCare.gov} project (\cite{Mergel2016}). All these properties make the waterfall approach rather unsuited for large projects, whose nature rarely turns out to be truly linear and sequential in reality (\cite{Kannan2014}).

Agile development approaches, in contrast, involve creating, testing, and improving technology products incrementally (\cite{Mergel2016}). The first example of such an approach dates back to the 1920s, when physicist and statistician Walter Shewhart of the Bell Labs developed the Plan Do Study Act (PDSA) to iteratively improve products and processes by responding to changes and new findings throughout the whole development process (\cite{Goldman1994}). Currently used agile methods include Scrum, Kanban, eXtreme programming (XP), Lean Software Development, and Feature-Driven Development (FDD) (\cite{Dingsoyr2012, Rigby2016a}). Design Thinking is often seen as a useful complement to further foster the user-centredness of the solution – although, strictly speaking, the method goes beyond the core of agile development methods (\cite{DaSilva2011, AustralianDigitalTransformationAgency2019}).

\begin{table}[ht!]
	\centering
	\begin{spacing}{.8}
	\caption{Values of agile software development}\label{tab: Values of agile software development}
	\renewcommand{\arraystretch}{1.4}
	\begin{tabular}{ p{1.5cm} p{9.5cm} }
		\hline
		\textit{Value 1} & Individuals and interactions over processes and tools\\
		\textit{Value 2} & Working software over comprehensive documentation\\
		\textit{Value 3} & Customer collaboration over contract negotiation\\
		\textit{Value 4} & Responding to change over following a plan\\
		\hline
		\multicolumn{2}{r}{Source: \cite{AgileManifesto2001}.}	
	\end{tabular}
    \end{spacing}
\end{table}

By and large, all agile approaches adhere to the tenets of the Agile Manifesto (\cite*{AgileManifesto2001}), which was formulated by a group of 17 thought-leaders in 2001 (\cite{Rigby2016a}). Table~\ref{tab: Values of agile software development} depicts the core values of this manifesto (\cite{Dingsoyr2012}). The first value highlights the importance of the people in the development team as well as their continuous interaction and communication. Valuing this over a pedantic obsession with processes and tools is supposed to make the team more responsive to sudden changes. Value number two emphasizes the adoption of a lean mentality when it comes to minimizing unnecessary work, in particular with regards to wasteful documentation procedures. This does not mean eliminating the documentation completely, but streamlining it in a way that provides the developer what is needed to do the work without getting caught up with unnecessary details (\cite{Eby2016}). The third value stresses the importance of collaborating with users and stakeholders vis-à-vis formal contract negotiations. This way it is supposedly ensured that their needs and concerns are truly understood and addressed by the development team. Finally, the fourth value acknowledges that uncertainties are a constitutive part of any software development project, and thus the team should stay flexible throughout the process to make changes.

\begin{wrapfigure}[8]{r}{0.6\textwidth}
	\centering
	\includegraphics[width=0.6\textwidth]{{"Latex/THESIS/Figures/Agile"}.pdf}
	\caption[Agile development process]{Agile development process (Author's visualisation)}
	\label{fig:Agile development process}
\end{wrapfigure}

To achieve this high level of responsiveness to change, the agile development process is made up of continuous iterations, so called "sprint cycles" (see Figure~\ref{fig:Agile development process}). At the beginning of an agile project the team conducts qualitative field research – potentially using Design Thinking methods – to gather the needs of the users and document them in so called "user stories" in the language of the users in order to avoid misconception and translation errors (\cite{Wirdemann2017, Mergel2016}). In each of the sprint cycles the project team then develops increments of working software that solve these user needs and are the pieces of the puzzle that ultimately make up the final service or product \parencite{Schwaber2020}. Each of these sprints usually takes between one to four weeks and, in principle, follows the phases of the traditional waterfall approach, however, with a strong focus on iteratively testing the increment (\cite{Schwaber2020, Hughes2013}). Small prototypes are used to rapidly (dis-)prove ideas before many resources are invested in their further development, following the motto of failing fast, early and often (\cite{Mergel2016, Boehmer2017}). The initial project plan serves as a rough guidance. Yet, the early evaluations reveal which requirements might not have been completely fulfilled and need to be added as work packages to the next sprint cycles, leading to a revision of the plan after each iteration (\cite{Mergel2016}). In consequence, the short sprint cycles paired with the continuous testing of the product or service are supposed to reduce the monetary risk involved in software projects \parencite{Schwaber2020}.

In a typical agile project, the development team is self-organising in an environment that enhances creativity and spontaneity. According to Takeuchi, one of the pioneering scholars of agile methods, and Nonaka  (\cite*{TakeuchiI1986}), three conditions have to be met in order to foster self-organising teams similar to the ones in start-ups: First, \textit{autonomy} as to how exactly the team plans to meet the user needs and requirements with the service or product. Second, \textit{self-transcendence} by establishing their own goals and amending them continuously based on user feedback as the service or product takes shape. Third, \textit{cross-fertilization} achieved through an interdisciplinary, cross-agency, and cross-functional team constellation that unites members with different skill-sets, thought process, and behaviour patterns. In addition to these three conditions the management has to exercise "subtle control" in order to enhance a team's creativity and spontaneity. This means establishing "enough checkpoints to prevent instability, ambiguity, and tension from turning into chaos", while at the same time avoiding the kind of rigid control that hinders the flow of the creative juices (\cite{TakeuchiI1986}). In most agile projects, the team's self-organisation is institutionalized by regular project meetings, as for example the daily stand-ups or the sprint planning and review meetings at the beginning and end of each sprint cycle in the case of the Scrum methodology (\cite{Scrum.org2019}). The 12 principles of the Agile Manifesto (\cite*{AgileManifesto2001}) represented in Table~\ref{tab: Principles of agile software development} summarize the aforementioned particularities of agile methods in a comprehensive manner.\footnote{The words "customer" and "business people" in the wording of the principles hint at the methodology's origin in the private sector domain. For a public sector set-up "customer" might be translated to "recipient of the government service or product" and "business people" might be translated to "public manager".}
\begingroup
\begin{spacing}{.8}
\renewcommand
\arraystretch{1.4}
\begin{longtable}[ht!]{p{0.14\textwidth} p{0.80\textwidth}}
\caption{Principles of agile software development}\label{tab: Principles of agile software development}
\renewcommand{\arraystretch}{1.4}\\
\hline
\textit{Principle 1}	& Our highest priority is to satisfy the customer through early and continuous delivery of valuable software.\\
\textit{Principle 2}	& Welcome changing requirements, even late in development. Agile processes harness change for the customer's competitive advantage.\\
\textit{Principle 3} &	Deliver working software frequently, from a
couple of weeks to a couple of months, with a preference to the shorter timescale.\\
\textit{Principle 4}	& Business people and developers must work
together daily throughout the project.\\
\textit{Principle 5}	& Build projects around motivated individuals.
Give them the environment and support they need, and trust them to get the job done.
\\
\textit{Principle 6}	& The most efficient and effective method of
conveying information to and within a development team is face-to-face conversation.
\\
\textit{Principle 7}	& Working software is the primary measure of progress.\\
\textit{Principle 8}	& Agile processes promote sustainable development. The sponsors, developers, and users should be able to maintain a constant pace indefinitely.
\\
\textit{Principle 9}	& Continuous attention to technical excellence
and good design enhances agility.\\
\textit{Principle 10 }& Simplicity – the art of maximizing the amount of work not done – is essential.\\
\textit{Principle 11} & The best architectures, requirements, and designs emerge from self-organising teams.
\\
\textit{Principle 12} & At regular intervals, the team reflects on how to become more effective, then tunes and adjusts its behaviour accordingly.\\
\hline
\multicolumn{2}{r}{Source: \cite{AgileManifesto2001}.}	
\end{longtable}
\end{spacing}
\endgroup

\subsection{Benefits, Challenges, \& Conditions}\label{Benefits, Challenges, and Conditions}
It has to be recognized that – besides all the potential benefits the approach can bring~– going agile is definitely not a panacea under all circumstances \parencite{Rigby2016}. Furthermore, the approach can even pose certain challenges to an organisation, in particular when it comes to applying it in government institutions whose values, structures, and processes might oftentimes be especially hard to combine with the agile spirit \parencite{Hajjdiab2011}.

On the one hand, literature has identified various potential \textit{benefits} that agile approaches can bring to government institutions. The most mentioned ones are as follows: First and foremost, scholars have found that agile methods lead to a better and more collaborative integration of IT teams with public sector management \parencite{Upender2005,berger2005uk, Berger2007,Dubinsky2005}.\footnote{As Sørensen and Torfing \parencite*{Sorensen2011} have presented in their well received article, collaboration in turn can lead to an enhancement of public sector (digital) innovativeness.} Second, agile methods are found to increase the satisfaction of government services' recipients and other important stakeholders \parencite{Fruhling2008,  Fulgham2011, iliev2009case}. Third, researchers have observed that an agile approach in government projects reduces the time required for value delivery \parencite{Upender2005, Surdu2006, McMahon2006, Moore2001}. Fourth and finally, agile approaches were found to improve team morale and increase public servants' job satisfaction \parencite{Hajjdiab2011, Vacari2015, Dubinsky2005}. 

On the other hand, research shows that applying agile approaches in public sector institutions can pose a number of \textit{challenges}: Most important in that respect is the often perceived cultural mismatch between the rigidity of bureaucratic line organisations and the flexibility of agile methods \parencite{Mergel, Fruhling2008, Altukhova2016, berger2005uk, Greve2019}. Second, since agile methods still constitute a novelty in public sector domains, the lack of education and experience in their application is another common challenge identified by scholars \parencite{Nuottila2016, McMahon2006, Fridman2016}. Third, the shortage of (senior) management support can also pose difficulties to agile projects in the public sector \parencite{Mergel,Berger2007,Dubinsky2005, Hajjdiab2011}. Fourth and finally, the necessary adaption of procurement and contracting arrangements as well as the compliance of the agile working mode with relevant standards and regulations can be a difficult task for public sector institutions that intend to go agile \parencite{Nuottila2016,Mergel, Fruhling2008, Fulgham2011}.

Ultimately, it is important to acknowledge which are the \textit{conditions} that are most likely favourable for applying agile methods, and under which circumstances the methodology is unlikely to yield benefits. In their Harvard Business Review article "Embracing Agile", Rigby, Sutherland, and Takeuchi \parencite*{Rigby2016} have stated various factors that are important in that respect. Adapted for public sector organisations, these factors are summarized in Table~\ref{tab:Favourable and unvarourable}. 

\begingroup
\begin{spacing}{.9}
\renewcommand
\arraystretch{1.5}
\begin{longtable}[ht!]{p{0.16\textwidth} p{0.38\textwidth}p{0.38\textwidth}}
	\caption{Conditions for applying agile methods in public institutions}\label{tab:Favourable and unvarourable}\\
	\hline
    \textbf{\textit{Conditions}} & \textbf{Favourable} & \textbf{Unfavourable} \\
    \hline
    \textit{Environment} & User preferences and solution options change frequently. & Conditions are stable and predictable. \\
    \textit{User \newline Involvement} & Close collaboration and rapid feedback are feasible. \newline Users know better what they want as the process progresses. & Requirements are clear at the outset and will remain stable. \newline Users are unavailable for constant collaboration. \\
    \textit{Innovation Type} & Problems are complex, solutions are unknown, and the scope is not clearly defined. Product specifications may change. Creative breakthroughs and delivery time are important. \newline Cross-functional collaboration is vital. & Similar work has been done before, and innovators believe the solutions are clear. Detailed specifications and work plans can be forecast with confidence and should be adhered to. Problems can be solved sequentially in functional silos.\\
    \textit{Modularity \newline of Work} & Incremental developments have value, and users can utilise them. Work can be broken into parts and conducted in rapid, iterative cycles. \newline Late changes are manageable. & Users cannot start testing parts of the product until everything is complete. \newline Late changes are expensive or impossible. \\
    \textit{Impact \newline of Interim \newline Mistakes} & They provide valuable learning. & They may be catastrophic. \\
    \hline
    \multicolumn{3}{r}{Source: Adapted from \cite{Rigby2016}}
\end{longtable}
\end{spacing}
\endgroup
\vspace{-0.1cm}
\subsection{Research Gap}\label{Research Gap}
Due to its growing relevance for practitioners, the application of agile methods in the public sector has increasingly become the focus of researchers around the world. However, recent literature reviews suggest that these studies are still limited in number and scope – most of the scholars employ case studies, yet, few also make use of surveys or interviews to derive their insights (\cite{CarvalhoFernandes2016, Vacari2015, Mergel2018, Mergel}). While all these studies uncover useful insights on the micro / organisational level, macro level studies that quantitatively analyse the evolution and spread of agile methods in government institutions are still lacking. This paper aims to help closing this research gap on the basis of an intra- and inter- country level comparison for the United Kingdom and Germany. 


\vspace{12cm}

\begin{center}
\tiny{\textit{(This white space is to illustrate how big the research gap actually is.)}}
\end{center}