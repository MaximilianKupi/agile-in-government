\section{Analysis}\label{Analysis}
\subsection{General Level}\label{General Level}
%
\begin{wrapfigure}{r}{0.5\textwidth}
    \centering
	 \includegraphics[width=0.48\textwidth]{{"Code/3_Data_Analysis/visualisations/digital_percentage"}.pdf}
	 \setlength{\belowcaptionskip}{-10pt}	\caption[Percentage of agil* web publications referring to digitalisation in Germany and the UK]{Percentage of agil* web publications referring to digitalisation in Germany and the UK}
	 \label{fig:digital_percentage}
\end{wrapfigure}
% 
On a general, cross-national level, it is to be noted, that the vast majority of crawler yielded sites mentioning agile working methods, also mention digital transformation related keywords (see Figure~\ref{fig:digital_percentage}). This is in line with the apparent connection between agile methods and the digital transformation of the public sector, as already discussed in Section~\ref{Digital Transformation}. 

Furthermore, as Figure~\ref{fig:levels and organisation} depicts, the number of crawled sites mentioning agile methods published on German and British government web presences has seen an increasing trend over the years, with a particular steep growth since 2017 (289 \%). Also, the number of different government \mbox{(sub-)}domains publishing agil* sites has been increasing over time. \begin{wrapfigure}[8]{l}{0.5\textwidth}
    \vspace{-11pt}
    \centering
	 \includegraphics[width=0.48\textwidth]{{"Code/3_Data_Analysis/visualisations/number_of_sites_and_organisations_over_time"}.pdf}
	 \setlength{\belowcaptionskip}{-18pt}
	 \caption[Agil* sites published and number of publishing domains over time in Germany and the UK]{Agil* sites published and number of publishing domains over time  in Germany and the UK}
	 \label{fig:levels and organisation}
\end{wrapfigure}
Both these developments can potentially be interpreted as hinting at the rising relevance of these methods for an increasing number of government organisations in Germany and Britain, and thus can be seen as a further confirmation of the themes discussed in current topic related public management literature (see Section~\ref{Literature Review}). 

\subsection{Intra-Country Comparison}\label{Intra-Country Comparison}
\subsubsection{Germany} 
The crawler yielded a total of 71 agil* sites published by German state and federal ministries.\footnote{For the numbers per state / ministry see Table~\ref{tab:Number of Sites} in \hyperref[Appendix C]{Appendix C}.} As becomes apparent in Figure~\ref{fig:German levels over time} (see next page), German government ministries, have only started publishing agile related content in 2015.\footnote{All stacked and vertical bar or point plots (see \hyperref[Appendix A]{Appendix A}) have the total number of agil* sites published by the respective institutions in brackets after the institution name.} Furthermore, 
\begin{wrapfigure}[14]{l}{0.5\textwidth}
    \centering
	 \includegraphics[width=0.5\textwidth]{{"Code/3_Data_Analysis/visualisations/german_levels_over_time"}.pdf}
	 \caption[Agil* sites published by German state and federal ministries over time]{Agil* sites published by German state and federal ministries over time}
	 \label{fig:German levels over time}	
\end{wrapfigure}
it can be seen, that federal level ministries – and amongst them the Federal Ministry of Labour and Social Affairs (13 sites, see Figure~\ref{fig:German ministries over time}) – have been the most active in that respect (23 sites in total), and the first ones to publish agile related content in 2015. The \href{https://www.bundesregierung.de/breg-de/themen/forschung/agil-arbeiten-bei-der-software-entwicklung-276792}{first site published} is an article of the Federal Government Office (\href{https://www.bundesregierung.de}{bundesregierung.de}) introducing agile as a new working method in software development and a topic that research relating to Germany's \href{https://www.hightech-strategie.de/de/hightech-strategie-2025-1726.html}{"High Tech Strategy 2025"} needs to pay close attention to as part of the future field innovations of the working world.\footnote{Link: \url{https://www.bundesregierung.de/breg-de/themen/forschung/agil-arbeiten-bei-der-software-entwicklung-276792}.} Analysing the crawler yielded sites in Germany makes apparent that the majority of them discuss or mention agile in a similar manner – as something that is (still) seen as mostly "external" to public administration and predominantly relevant for the private sector only. Due to this one article, the 
\begin{wrapfigure}{r}{0.5\textwidth}
	\centering
	 \includegraphics[width=0.5\textwidth]{{"Code/3_Data_Analysis/visualisations/germany_ministries"}.pdf}
	 \caption[Agil* sites published by German federal ministries over time]{Agil* sites published by German federal ministries over time}
	 \label{fig:German ministries over time}
\end{wrapfigure}

\noindent Federal Government Office also ranks highest in terms of average number of co-search terms (e.g. 'scrum', 'sprint', 'design thinking'; see Table~\ref{tab:Co search terms} 
in \hyperref[Appendix A]{Appendix A}) with an average of 9 co-search terms per site (see Figure~\ref{fig:German agile depth federal and state} on next page).\footnote{However, this average number might be skewed, since the crawler only yielded one article published by the Federal Government Office.} The number of co-search terms might be read as a rough proxy of how in-depth the agile methodology is being described on the respective site. 

When comparing the content published by the ministerial web domains of the federal states (see Figure~\ref{fig:map} next page), it \begin{wrapfigure}[8]{r}{0.5\textwidth}
    \centering
	 \includegraphics[width=0.5\textwidth]{{"Code/3_Data_Analysis/visualisations/average_number_of_co_terms_german_organisations"}.pdf}
	 \caption[Average number of co-search terms per agil* site published by German federal ministries and state governments]{Average number of co-search terms per agil* site published by German federal ministries and state governments}
	 \label{fig:German agile depth federal and state}
\end{wrapfigure}
\FloatBarrier 
\noindent 
can be seen that Baden-Wuerttemberg's ministries have published the most number of websites referring to agile (9 pages), followed by Berlin (8), Bremen, and Rhineland-Palatinate (both 6). However, in terms of how many co-search terms appear on average on those sites, Bavaria leads the board with an average of 4.2 co-search terms per site, followed by Hesse (3.25), and Brandenburg (3). In Bavaria's case the high average number of co-search terms is due to two articles of "Elite Network of Bavaria" – an initiative of the Bavarian State Ministry of Science and the Arts – which describe the methodology in length as part of their educational offers.\footnote{Links: \url{https://www.elitenetzwerk.bayern.de/elitestudiengaenge/aktuelles-esg/artikel0/scrum-workshop-bei-hubert-burda-media-433/?L=248}, and \url{https://www.elitenetzwerk.bayern.de/elitestudiengaenge/aktuelles-esg/artikel0/agile-transformation-412/}.} 
\begin{figure}[ht]
	\centering
	\begin{tabular}{c c}
    Number of Agil* Sites Published & \makecell{Average Number of Co-Search \\ Terms per Site} \\
	\includegraphics[width=0.4\textwidth]{{"Code/3_Data_Analysis/visualisations/map_breadth_export"}.pdf} &
	\includegraphics[width=0.4\textwidth]{{"Code/3_Data_Analysis/visualisations/map_depth_export"}.pdf}
	\end{tabular}
	\setlength{\belowcaptionskip}{-10pt}
	\caption[Breadth and depth of German federal state ministries' web publications referring to agil*]{Breadth and depth of German federal state ministries' web publications referring to agil*}
	\label{fig:map}
\end{figure}
\noindent Again, these articles are not related to agile methods in public administration per se. Hesse's high score, however, is due to two pages talking about the methodology in the context of introducing an innovation laboratory for staff of the regional council Kassel as well as one page mentioning agile methods with respect to the digital transformation of the public administration.\footnote{Links: \url{https://rp-kassel.hessen.de/forumzukunft}, \url{https://rp-kassel.hessen.de/\%C3\%BCber-uns/stabsstelle-forum-zukunft/inlab/innovationslabor}, and \url{https://hzd.hessen.de/leistungen/digitalisierung-erfolgreich-gestalten}.} Hence, a direct link between the methods and government institutions is established in Hesse's case. The scraper did not yield any matching pages for Saarland, North Rhine-Westphalia, Saxony, and Lower Saxony.

Thus, as a final, summarising remark on Germany's intra-country comparison, it has to be stated that there are substantial differences between states and between ministries when it comes to the number of agil* sites published as well as the average number of co-search terms appearing on these sites. In recent years, the Federal Ministry of Labour and Social Affairs, as well as the states of Baden-Wuerttemberg, Berlin, Bavaria, and Hesse have turned out to be most active in that respect (see Figures~\ref{fig:German ministries over time} and \ref{fig:German agile depth federal and state} in \hyperref[Appendix A]{Appendix A}). The average differences between central and federal state institutions, however, are rather minor, hinting at a somewhat federalistic approach to bringing agile into government.

\subsubsection{United Kingdom} 
Looking at the matching sites that the scraper yielded from 'gov.uk' – all in all a total of 387 pages\footnote{144 of these pages (37.21\%) originate from gov.uk's \href{https://webarchive.nationalarchives.gov.uk/search/}{national web archive}, where outdated and deleted web sites are preserved.} – reveals the following patterns over time (see Figure~\ref{fig:British organisations over time} on next page).\footnote{Although the initial goal of the paper was to only include highest government bodies in the analysis (i.e. cabinet office and ministries), crawling gov.uk as the single government domain for Britain also yielded results of subordinate authorities. In order not to discard this data, and potentially also to uncover relevant insights on the topic, these sites were also included in the intra-country comparison of Britain.} The very first content on agile methods with the fitting title 'Opening up the conversation' was published in 2011 by Directgov's development team.\footnote{For a distinction between all British central government ministerial departments over time see Figure~\ref{fig:British ministries over time} in \hyperref[Appendix A]{Appendix A}. Grouping of institutions according to \url{https://www.gov.uk/government/organisations}.} It is \href{https://webarchive.nationalarchives.gov.uk/20110410092517/http://innovate.direct.gov.uk/
}{the start page of innovate.direct.gov.uk}, a former innovation platform of direct.gov.uk – UK's official website back then – to share "ideas for digital development [...], propose ideas for new projects and keep up to date with our [the development team's] latest work".\footnote{Link: \url{https://webarchive.nationalarchives.gov.uk/20110410092517/http://innovate.direct.gov.uk/
}.}\footnote{To a certain extent, this development team could be viewed as a predecessor to Government Digital Service.} Interestingly, the second publication was about how Britain's \href{https://webarchive.nationalarchives.gov.uk/20110406081839/http://www.cabinetoffice.gov.uk/content/government-ict-strategy}{new ICT strategy} with a focus on adopting agile methods will help "cutting public spending on Government information and communication technology (ICT) by millions of pounds" – shedding light on a potential motivation for the British government to go agile.\footnote{Link: \url{https://webarchive.nationalarchives.gov.uk/20110406074635/http://www.cabinetoffice.gov.uk/news/government-save-millions-ict}.} In line with that, other publications at the beginning of 2011 are the \href{https://webarchive.nationalarchives.gov.uk/20110406081839/http://www.cabinetoffice.gov.uk/content/government-ict-strategy}{ICT Strategy} itself as well as announcements of \href{https://webarchive.nationalarchives.gov.uk/20110803134807/http://www.cabinetoffice.gov.uk/news/joe-harley-cbe-appointed-chief-information-officer-uk-government}{a change of personnel} and \href{https://webarchive.nationalarchives.gov.uk/20110406074527/http://digitalengagement.cabinetoffice.gov.uk/blog/2011/03/29/towards-a-single-government-domain/}{the plan to develop "a single government domain" (gov.uk)}.\footnote{Links: \url{https://webarchive.nationalarchives.gov.uk/20110406081839/http://www.cabinetoffice.gov.uk/content/government-ict-strategy}, \url{https://webarchive.nationalarchives.gov.uk/20110803134807/http://www.cabinetoffice.gov.uk/news/joe-harley-cbe-appointed-chief-information-officer-uk-government}, and \url{https://webarchive.nationalarchives.gov.uk/20110406074527/http://digitalengagement.cabinetoffice.gov.uk/blog/2011/03/29/towards-a-single-government-domain/}.} Besides Directgov and the Cabinet Office, the only other institution publishing agile methods related content in 2011 was Government Digital Service. This "centre of excellence" was founded in 2011 by the Cabinet Office and commissioned to take the leading role in the British government's digital transformation (\cite{GovernmentDigitalService2020}). Consequently, their first crawler yielded publications have been about \href{https://webarchive.nationalarchives.gov.uk/20120404152514/http://digital.cabinetoffice.gov.uk/2011/10/25/the-second-lever/}{them wanting to hire "world-class digital talent"}, \href{https://webarchive.nationalarchives.gov.uk/20120404161123/http://digital.cabinetoffice.gov.uk/2011/12/08/new-home-for-gds/}{moving into their new office} or \href{https://webarchive.nationalarchives.gov.uk/20121102172608/http://digital.cabinetoffice.gov.uk/2011/09/19/introducing-the-needotron-working-out-the-shape-of-the-product/}{introducing specific tools applied for agile development}.\footnote{Links: \url{https://webarchive.nationalarchives.gov.uk/20120404161123/http://digital.cabinetoffice.gov.uk/2011/12/08/new-home-for-gds/}, \url{https://webarchive.nationalarchives.gov.uk/20120404152514/http://digital.cabinetoffice.gov.uk/2011/10/25/the-second-lever/}, and \url{https://webarchive.nationalarchives.gov.uk/20121102172608/http://digital.cabinetoffice.gov.uk/2011/09/19/introducing-the-needotron-working-out-the-shape-of-the-product/}} 
\begin{figure}[ht!]
	\centering
	 \includegraphics[width=1\textwidth]{{"Code/3_Data_Analysis/visualisations/british_organisations_over_time"}.pdf}
	 \setlength{\belowcaptionskip}{-10pt}
	 \caption[Agil* sites published over time on gov.uk, distinguished between organisations]{Agil* sites published over time on gov.uk, distinguished between organisations}
	 \label{fig:British organisations over time}
\end{figure}

It is to be noted that both, the Cabinet Office and Government Digital Service, are the only institutions on gov.uk consistently publishing agile related content every year since 2011 as well as the ones with the highest number of sites published (a total of 55 and 145 sites respectively). In terms of average number of co-search terms, the Home Office leads the ranking with 7.34 co-search terms per site, followed by the Education \& Skills Funding Agency (7), and the Department for Environment, Food 
\& and Rural Affairs (6) (see Figure~\ref{fig:British agile depth organisation} in \hyperref[Appendix A]{Appendix A}).\footnote{However, these results might be slightly skewed as the crawler only yielded few pages for the respective institutions (a total of 3, 2, and 4 pages respectively).}

Over time, the crawled sites show a similar pattern than in the case of Germany: The number of organisations publishing content on agile methods generally increases. However, non-central government agencies in Britain are (still) significantly less active in that respect – the councils of Durham, Essex, Leicester and North Yorkshire as well as the Greater London Authority and the Local Government Authority published only one site each. Also, it seems like the election cycle might be somewhat correlated with the number of published sites: One year after the general elections (in 2015 and 2017) there is a notable increase in sites published.\footnote{The first notable increase in 2013 could potentially also be linked to the general elections in 2010, however the effects are only noticeable with a delay, due to Government Digital Service needing to ramp up their work. Congruently to the central government's behaviour, local authorities also only started publishing agile related content after the local elections in 2018. Additionally, the spread of the methodology to the regional government might be explained by government experts from the field moving to the countryside in their mid-thirties/forties to have better "family life conditions", after gaining first work experience at GDS.} The \href{https://www.gov.uk/service-manual/communities/agile-delivery-community}{Agile Delivery Community} clearly was the leading institutions responsible for the spike in 2016. It is a community of practice to inform about, share experiences related to, and discuss agile ways of working, and hence might be seen as a potential vehicle to roll out the methodology to other government institutions.\footnote{Link: \url{https://www.gov.uk/service-manual/communities/agile-delivery-community}} In line with that assumption, the number of different organisations publishing agil* sites shows a significant growth pattern since 2016. Also, it is noteworthy that 42 percent of the agile content was published on online blogs of the respective organisations.\footnote{The percentage was calculated by comparing blog sub-domains, e.g. "blog.gov.uk", with non-blog sub-domains, e.g. "gov.uk".} Thus, it seems as if this medium has a special role to play in the information and discussion about novel working approaches – potentially, because a blog is a low key, less formal information channel where government officials dare to publish content related to less "traditional" topics.

The final summarising remark for Britain's intra-country comparison is as follows: With Government Digital Service and the Cabinet Office there are clearly identifiable lead innovators with respect to the number of agil* sites published and the average number of co-search terms appearing on those sites. Other institutions' efforts in that respect are substantially lesser. In particular non-central government institutions (i.e. local authorities) started to publish agile content much later and with less intensity. This hints at a rather centralistic approach to introducing agile methods into government.
\newpage
\subsection{Inter-Country Comparison}\label{Inter-Country Comparison} 
\begin{wrapfigure}[10]{r}{0.5\textwidth}
	\centering
	 \includegraphics[width=0.45\textwidth]{{"Code/3_Data_Analysis/visualisations/british_vs_German_federal_ministries_over_time_without_archive"}.pdf}
	 \setlength{\belowcaptionskip}{-30pt}
	 \caption[Agil* sites published by British and German ministries over time]{Agil* sites published by British and German ministries over time}
	 \label{fig:Agil* sites published over time by British and German ministries over time}
\end{wrapfigure}
In order to have the most legitimate basis for a quantitative comparison, only the (federal level) ministerial departments and the Cabinet / Federal Government Office are included in the following inter-country comparison.\footnote{The two countries' state systems are rather distinct – Britain is a unitary state, while Germany is federal (\cite{Elazar1997}) – and hence, including non-central state institutions into the comparison was not reasonable. Furthermore, the crawl on the single government domain gov.uk also yielded results from subordinate agencies, which therefore were included in the intra-country comparison of Britain. For Germany, however, only the highest government institutions (i.e. cabinet offices and ministries) were crawled, since an in- or (unintended) exclusion of any further subordinate institution's web site as seed URL would have potentially introduced a human selection bias. Therefore, subordinate institutions were also excluded from the comparison. Finally, 22 of the British ministry sites – the majority of sites before 2015 – stem from gov.uk's \href{https://webarchive.nationalarchives.gov.uk/search/}{national web archive}. Germany does not have a comparable archive, and hence outdated and deleted German ministry pages mentioning agile methods might have simply been lost, potentially skewing the comparison in Britain's favour. Therefore, all British sites stemming from the web archive have been excluded in this analysis. As expected, including those pages further widens the lead of British ministries (see Figure
~\ref{fig:Agil* sites published over time by British and German ministries over time WITH ARCHIVE} and \ref{fig:Breadth and depth of agil* sites published by British and German ministries (with ministry names) WITH ARCHIVE} in \hyperref[Appendix A]{Appendix A}).} 

Figure~\ref{fig:Agil* sites published over time by British and German ministries over time} depicts the wide lead – both, in quantity and timing – that British ministries have over German ones in terms of agil* sites published: British ministries started publishing agile content three years earlier and in total published 3.6 times more pages than German ministries (86 British vs. 24 German sites).

As can be seen in Figure~\ref{fig:Breadth and depth of agil* sites published by British and German ministries} (see next page), the most active British ministerial department in this respect – as well as by far the most active in general – is the Cabinet Office (37 published sites in total), while in Germany it is the Federal Ministry of Labour and Social Affairs (13 published sites).\footnote{See Figure~\ref{fig:German ministries over time} and Figure~\ref{fig:British ministries over time WITHOUT ARCHIVE} in \hyperref[Appendix A]{Appendix A} for a depiction over time. Furthermore, see Figure
~\ref{fig:Breadth and depth of agil* sites published by British and German ministries (with ministry names)} for a depiction of Figure~\ref{fig:Breadth and depth of agil* sites published by British and German ministries} including the complete ministry names.} In terms of highest number of average co-search terms per site, the German Federal Government Office (9 co-search terms on average) beats the British Home Office (7.33 terms). Yet, the crawler only yielded one page (German Federal Government) and three pages (British Home Office) for these institutions, and thus this high number might be skewed. Over all, it has to be noted that the total number of ministries publishing agile related content on their web presences is more than double for Britain (13 British vs. 6 German ministries), and the total average of agil* sites published (6.62 vs. 4 sites) as well as the total average number of co-search terms per site (3.87 vs. 2.66) is considerably higher for British ministries – see neon yellow markers in Figure~\ref{fig:Breadth and depth of agil* sites published by British and German ministries}).\footnote{As can be seen by the substantially lower respective numbers for other German ministries, the 'high' total average number of co-search terms per site in the case of Germany is only driven by that one 'outlier' Federal Government Office site – also, because the total number of ministries publishing agile related content is lower and the respective numbers of the ministries have a stronger influence on the average. Hence, the difference between British and German ministries in that respect is actually larger than the difference in averages suggests.}

\begin{figure}[ht!]
	\centering
	 \includegraphics[width=1\textwidth]{{"Code/3_Data_Analysis/visualisations/Comparison_Federal_Ministries_without_names_without_archive"}.pdf}
	 \caption[Breadth and depth of agil* sites published by British and German ministries]{Breadth and depth of agil* sites published by British and German ministries}
	 \setlength{\belowcaptionskip}{-100pt}
	 \label{fig:Breadth and depth of agil* sites published by British and German ministries}
\end{figure}

When comparing the top frequent co-search terms appearing on all published agile related sites between British and German ministries (see Figure~\ref{fig:wordfrequencies} on next page), it becomes apparent that British ministries' content focuses way more on the user and their needs while in Germany agile relates more to the rather superficial terms that describe a new agile way of working ("agil arbeit" \& "agile arbeitsformen") or a general innovation culture ("innovationskultur").\footnote{Since these are the word lemmas, 'agil arbeit' stands for 'agiles Arbeiten' (Eng.: 'agile ways of working'), 'agile Arbeitswelt' (Eng.: 'agile world of work')  For a depiction of the most frequent co-search terms as word clouds where the size of each word correlates with the number of times the co-search term appears on all agil* sites published by British and German (federal) ministries see Figure~\ref{fig:wordclouds} in \hyperref[Appendix A]{Appendix A}.} This is consistent with the rather "shallow" description of the topic by the German ministries with regard to the average number of co-search terms (see previous paragraph) as well as with the comparatively more skewed distribution of co-search terms: The most frequent term lemma 'agil arbeit' (Eng.: 'agile work') occurs more than three times more on the web(see Figure~\ref{fig:wordfrequencies}). It seems as if German ministries have not yet grasped the major difference when it comes to agile methods: The focus on users (i.e. citizens) and the fulfillment of their needs through thorough user research and good service design in iterative sprints (see Section~\ref{Agile Methods}). Much rather, they still seem to regard agile as a new, mostly non-public administration related trend in the world of work. The British co-search term frequencies, in contrast, suggest that the British ministries have done their homework when it comes to understanding the ins and outs of agile methods and presenting these on their web presences.


\begin{figure}[ht!]
    \begin{tabular}{c c}
    British Ministries & German Ministries\\
    \includegraphics[width=0.485\textwidth]{{"Code/3_Data_Analysis/visualisations/word_freq_uk"}.pdf} & \includegraphics[width=0.485\textwidth]{{Code/3_Data_Analysis/visualisations/word_freq_germany"}.pdf}
    \end{tabular}
%	\textbf{Test}\par\medskip
%	\includegraphics[height=0.35\textwidth]{{"Code/3_Data_Analysis/wordcloud_germany"}.pdf}
%	\includegraphics[height=0.35\textwidth]{{"Code/3_Data_Analysis/wordcloud_uk"}.pdf}
	\caption[Frequency of co-search term lemmas for all agil* sites published by British and German ministries]{Frequency of co-search term lemmas for all agil* sites published by British and German ministries\footnotemark}
	\label{fig:wordfrequencies}
\end{figure}
\footnotetext{For an English translation of the German co-search terms see Table~\ref{tab:Co search terms} in \hyperref[Appendix A]{Appendix A}.}

