\section{Discussion}
\subsection{Summary of Findings}
%This paper examined the evolution and spread of agile governance methods and principles in British and German public administrations. Summing up the analysis results from Section~\ref{Analysis}, the major findings on a general level as well as with respect to an intra- and inter-country comparison are as follows. 
%On a general level, it became apparent that the importance of agile methods for British and German government institutions has been increasing over time. Particularly in recent years, the number of agil* sites published as well as the number of domains publishing 

%MAKE A SUMMARY TABLE

\subsection{Policy Implications}
\subsection{Future Work}
%The limitations of the paper come with suggestions for future research. 
% brittain has a lot of content on blogs --> good outlet to discuss new developments?

% clear strategy (digital strategy) helps?

% no webarchives in germany --> hence, probably lesser hits

% brittain digitalisation from the outside in --> digital as a tool to more efficient and effectively reach fulfill citizens needs (refer back to first ICT strategy article) 

% from mehmet
%findings summary one short paragraph
%contributions and implications for research one paragraph
%implications / recommendation for practice two paragraphs
%limitations of the study one or two paragraphs

%limitations: 
%   no pdfs included in the search algorithm
%   checking each cleaning step and the whole error of all crawls (due to time limits not possible this time) 
% The seed URL for Britain was set to gov.uk, which basically includes the web presences of all British government institutions and agencies. However, for Germany, which does not have a single government domain, only the ministerial websites have been set as seed URLs, thus leading to an exclusion of potential subordinate authorities like the IT provider for the federal government, \href{https://www.itzbund.de/DE/Home/home_node.html}{ITZBund}. Expand the crawl also to subordinate authorities in germany


% blogs as good ways to discuss and try out innovative forms of governance

% research contributions
% shows the evlution and spread of agile methods in relevant parts of the british and German public sector
% new way of doing social science research (venturing into new ways of...)

% potentially do a writers analysis to find out who has been writing those articles (is it only a hand full of people maybe?)

% blog as low key information channel
% werden noch stiefmütterlich behandelt

%nur zwei blogs in deutschland
%https://mb.sachsen-anhalt.de/start/blog-detailansicht/?cHash=76b9e1e8417e5e36ae5105ad14afa4f2&tx_t3extblog_blogsystem%5Baction%5D=show&tx_t3extblog_blogsystem%5Bday%5D=23&tx_t3extblog_blogsystem%5Bmonth%5D=06&tx_t3extblog_blogsystem%5Bpost%5D=267&tx_t3extblog_blogsystem%5Byear%5D=2018


%https://frauenseiten.bremen.de/blog/internationales-sommerstudium-informatica-feminale/

% maybe brittain is more top down thus change from above while germany is more bottom up (change from below) --> need to check in future research

% potentially german government doesn't yet see digital as an integral part to delivering their services to the citizens (hence, digital related approaches are also less valued) 