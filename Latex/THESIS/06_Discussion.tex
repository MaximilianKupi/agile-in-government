\section{Discussion}\label{Discussion}
\subsection{Summary of Findings}\label{Summary of Findings}
This paper examined the evolution and spread of agile methods in British and German government institutions. To do so, the crawling results from government websites – in particular cabinet and ministry pages – have been analysed for the appearance of agile terms as well as other topic relevant co-search terms.\footnote{The average number of co-search terms gives an indication of how in-depth the topic is being discussed on the respective sites.} Drawing on the results from Section~\ref{Analysis}, the major findings on a general level as well as with respect to an intra- and inter-country comparison are summarized in Table~\ref{tab:Summary of findings}.

\begingroup
\begin{spacing}{.9}
\renewcommand
\arraystretch{1.73}
\begin{longtable}[ht!]{p{0.16\textwidth} p{0.38\textwidth} p{0.38\textwidth}}
	\caption{Summary of findings}\label{tab:Summary of findings}\\
	\hline
    \textbf{\textit{Level}}& \textbf{Germany} & \textbf{United Kingdom} \\
    \hline
    \textit{General} & \multicolumn{2}{p{0.79\textwidth}}{
    \begin{minipage}[t]{\linewidth}
    \begin{itemize}[nosep, wide=0pt, leftmargin=*, after=\strut]
    \item Number of agile related sites published by government institutions increased over time and particularly in recent years (289\% increase from 2017 to 2019).
    \item Most agil* sites published (84\%) also mention digital transformation related keywords.
    \item  Both observations are in line with theses in agile related public management literature (cf. e.g. \cite{Mergel}).
    \end{itemize}
    \end{minipage}}\\
    \textit{Intra-Country \newline (also including subordinate institutions as well as sites from gov.uk's web archive for Britain)}& 
    \begin{minipage}[t]{\linewidth}
    \begin{itemize}[nosep, wide=0pt, leftmargin=*, after=\strut]
    \item Agile is rather novel trend; first agil* sites published in 2015; 74 sites in total.
    \item Most active publisher is Federal Ministry of Labour \& Social Affairs (13 pp.) and the state of Baden-Wuerttemberg (9 pp.).
    \item Federal Government Office (9 terms/site) and Bavarian government (3.2 terms/site) leading in terms of average number of co-search terms per site.
    \item Average differences between central and non-central authorities with respect to timing and quantity of agil* sites published are minor.
    \end{itemize}
    \end{minipage} &
    \begin{minipage}[t]{\linewidth}
    \begin{itemize}[nosep, wide=0pt, leftmargin=*, after=\strut]
    \item Government institutions have been discussing agile methods on web presences since 2011; 380 sites in total. 
    \item Leading publishers are Government Digital Service (143 pp.) and Cabinet Office (55 pp.).
    \item Highest average number of co-search terms by Home Office (7.34 terms/site) and Education \& Skills Funding Agency (7 terms/site).
    \item Average differences between central and non-central, i.e. local authorities are major.
    \end{itemize}
    \end{minipage}\\
    \textit{Inter-Country \newline (only federal ministries and cabinet / federal government office)} &
    \multicolumn{2}{p{0.79\textwidth}}{
    \begin{minipage}[t]{\linewidth}
    \begin{itemize}[nosep, wide=0pt, leftmargin=*, after=\strut]
    \item British ministries started publishing agil* sites 3 years earlier and published 3.6 times more sites than German ministries (86 British vs. 24 German sites).
    \item Number of British ministries publishing agile related content is more than twice as high (13 British vs. 6 German ministries); they published 1.32 times more sites on average (6.62 British vs. 4 German sites on average); and have a 1.57 times higher total average number of co-search terms per site (3.87 British vs. 2.47 German total average). 
    \end{itemize}
    \end{minipage}} \\
    \textit{Inter-Country \newline (cont'd)} &
    \multicolumn{2}{p{0.79\textwidth}}{
    \begin{minipage}[t]{\linewidth}
    \begin{itemize}[nosep, wide=0pt, leftmargin=*, after=\strut]
    \item Most active British ministry (Cabinet office, 37 sites) has 3.1 times more published agil* sites than most active German ministry (Ministry of Labour \& Social Affairs, 12 sites).
    \item German central government institution with most in-depth description of method (Federal Government Office, 9 terms/site) has 1.23 times more co-search terms on average per site than respective British ministry (Home Office, 7.33 terms/site).
    \item While British ministries show great knowledge about particularities of agile methods (many co-search terms related users, iteration, and sprint), agile methods description on German ministries' web pages seems rather general and shallow (many co-search terms related to agile ways of working or innovation culture).
    \end{itemize}
    \end{minipage}}\\
    \hline
\end{longtable}
\end{spacing}
\endgroup
\vspace{-0.1cm}


The results can be read as sign for the increasing importance that agile methods hold for government institutions in the face of the digital transformation. Furthermore, they reveal the strong agile leadership position that British government institutions – in particular the ministerial organisations – have in comparison to Germany. Also, as mentioned in the \hyperref[Introduction]{Introduction}, Government Digital Service – the centre of excellence for the British government's digital transformation – in fact turned out to have paved the way. Yet, the results also highlight the important role of the British Cabinet Office. Finally, the rather minor average differences between central and non-central government institutions in the case of Germany potentially hint at a somewhat federalistic approach to bringing agile methodologies into government. For Britain, the substantial differences between central and local authorities with respect to timing and quantity of agil* sites published suppose a rather centralistic approach.\footnote{However, a direct quantitative comparison between Britain and Germany in that respect is not sensible since the German federal state institutions (still) own substantially more political leeway than British local authorities. (Despite attempts to improve the situation for British non-central bodies with the help of the Localism Act \parencite{Legislation.gov.uk2011}, the deep-rooted centralisation in Britain is hardly challenged (see for example Jeraj (\cite*{Jeraj2013}) and Pipe (\cite*{ Pipe2013}))).} Both these observations are in line with the unitary (Britain) or respectively federal (Germany) state system of each country \parencite{Elazar1997}. 

In addition to uncovering above findings, this work contributes to the literature by exemplifying how a 'webometrical' analysis approach \parencite{Thelwall2009} based on self-crawled web content can yield substantive insights for public management research. 

\subsection{Policy Implications}\label{Policy Implications}
The policy implications of this work can be distinguished between implications addressing British government officials, implications for German government officials, and implications addressing officials of both and potentially also further countries.

For British Government officials the implications are rather complimentary. Their work, in particular the continuously persistent work of the Cabinet Office and Government Digital Service, appears to have paid off. According to the respective web publications, agile methods are spreading amongst a wide range of government institutions with their understanding of the methodology's particularities seeming rather substantial. Consequently, British government officials should keep up this spirit in order to truly and fully become an agile, digitally transformed government. One thing they could consider though, potentially is strengthening local authorities in their application of agile methods. Since ultimately it is these government bodies that are responsible for providing a substantive set of vital services to the citizens \parencite{LocalGovernmentAssociation2020}, and hence might also benefit from a more agile approach.

To German government officials, on the other hand, one would like to say: "You are late for the agile party and seem to have a rather unusual understanding of the party's theme." To catch up, they could possibly adopt some of the strategies of their British colleagues: First, a clear acknowledgement that the digital transformation of government needs agile methods to reach its full potential would set the necessary stage for public administrations to become active and stop seeing agile as a mostly external phenomenon. Perhaps they could do this by making the link between agile methods and digital transformation much clearer in their strategy papers on the topic (see Section~\ref{Digital Transformation}). Second, they might consider having more distinct lighthouse institutions that internally push the application of agile methods and talk about it offensively so that other institutions take notice. Although the German Federal Ministry of Labour \& Social Affairs seems to have opted for this position recently, a clearer sign from above, i.e. the Federal Government Office, might turn out to be more impactful.\footnote{The latest efforts of the Federal Government Office (i.e. \href{https://tech.4germany.org/}{Tech} and \href{https://work.4germany.org/}{Work 4 Germany}) are a good start but seem rather lukewarm and with limited radiant power. Whether or not this "sign from above" should even be a "lead from the top" as in Britain's case, and if so, whether this would be compatible with German's federal power dynamics is a question for another research paper.} Third, they could establish more government related online blogs, which would provide a low key communication channel where government officials could share their experiences and knowledge on the topic.\footnote{So far, blogs are treated rather poorly by German government officials in that respect: Only two of the crawler yielded agil* sites, as opposed to more then half of the British sites, originated from government blogs (\href{https://mb.sachsen-anhalt.de/start/blog-detailansicht/?cHash=76b9e1e8417e5e36ae5105ad14afa4f2&tx_t3extblog_blogsystem\%5Baction\%5D=show&tx_t3extblog_blogsystem\%5Bday\%5D=23&tx_t3extblog_blogsystem\%5Bmonth\%5D=06&tx_t3extblog_blogsystem\%5Bpost\%5D=267&tx_t3extblog_blogsystem\%5Byear\%5D=2018}{Link to blog entry one} and \href{https://frauenseiten.bremen.de/blog/internationales-sommerstudium-informatica-feminale/}{link to blog entry two}, in addition both entries are only rather loosely related to government institutions).} Fourth, in line with that, establishing agile communities of practice can potentially provide a further useful platform for exchange. Fifth and finally, Germany could consider adopting a similar "outside-in-approach" as Britain: Their digital, and therewith agile transformation apparently strongly coincided with the implementation of the single government domain (gov.uk) by Government Digital Service (see Section~\ref{Analysis}). While this conversion potentially provided a much more efficient and effective way to reach out to citizens and fulfill their needs, it also made government institutions come into contact with Government Digital Service when transferring their web content to the new domain, and thereby might also have provided them with a first possibility to experience the advantages of using agile methods.\footnote{Interestingly, the most active German federal states in terms of number of agil* sites published and average number of co-search terms (Baden-Wuerttemberg, Bavaria, and Hesse) all have (mostly) single government domains (only two ministries from Baden-Wuerttemberg have their own domains).}

Finally, the policy implication for British as well as German government officials, and potentially also beyond is that agile methods apparently have become a considerable tool for managing projects related to government's digital transformation. Hence, further experimenting and honest best practice sharing will be needed in order to fully reap the methodology's potential in better serving the citizens' needs with digital services. 

\subsection{Limitations \& Future Work}\label{Limitations and Future Work}
This work is not without limitations; they come with suggestions for future research. First, the web crawl only included plain HTML web site content, and hence information from PDF, video or audio files for example was not analysed. Since, for certain government institutions uploading information in PDF format (still) might be common practice, future iterations should potentially try to find ways to also include these type of online sources. Second, due to time limitations, the whole process of data collection as well as all the design choices for data preprocessing were not analysed as thoroughly as they potentially could have been.\footnote{Besides, of course, all the analyses to avoid obviously identifiable errors.} Future work thus might dive deeper into the log files of the crawls to identify potential subtle irregularities that might impede the reliability of the crawled data or compare various data cleaning approaches in terms of their accuracy. Third, following this line of thought, future researchers could build their own crawler architecture from scratch to not rely on third party code (in this case the Scrapy package), and further tailor the crawler's behaviour to their research goal. Fourth, the composition and analysis of the co-search terms could potentially take a more data driven approach based on the gathered texts themselves, e.g. by analysing all words in a certain window around the agil* term, and not only rely upon manually predefined dictionaries. Fifth, the seed URL for Britain was set to gov.uk, which basically includes the web presences of all British government institutions and agencies. However, for Germany, which does not have a single government domain, only the ministerial websites have been set as seed URLs, thus leading to an exclusion of potential subordinate authorities like the IT provider for the federal government, \href{https://www.itzbund.de/DE/Home/home_node.html}{ITZBund}. Hence, in order to also yield insights on the role of such subordinate institutions and further expand the basis for a legitimate intra-country comparison, future research should include their respective web presences as additional seed URLs. Finally, since the crawler yielded pages can only serve as a proxy for the true status of the application of agile methods in the respective institutions, researchers could potentially triangulate these results with supplementary data sources, e.g. qualitative interviews, quantitative surveys, further online data etc. 

Additional research alleys that could be build on or amend the results of this paper but go beyond the core of its research question are as follows. Researchers could dive deeper into analysing the text content with author or plagiarism detection methods in order to find out which author(s) or institution(s) actually 'feed' the appearance of agile methods on government websites. Also, web sites from NGOs, consultancies or relevant media outlets could be included in the crawls to inspect the relation between these sites and government websites, and potentially find out who or what 'drives' the adoption of agile methods in the public sector. Furthermore, future work could examine how the differences in state system – federal Germany vs. unitary Britain – influence the adoption of agile methods or digital transformation in general. Finally, researchers could further explore how crawled online content can serve as reliable data source for social science research.