\section{Discussion}
\subsection{Summary of Findings}
This paper examined the evolution and spread of agile governance methods and principles in British and German government institutions. To do so, the crawling results from government websites – in particular cabinet and ministry pages – have been analysed for the appearance of agile terms as well as other topic relevant co-search terms.\footnote{The average number of co-search terms gives an indication of how in-depth the topic is being discussed on the respective sites.} Drawing on the results from Section~\ref{Analysis}, the major findings on a general level as well as with respect to an intra- and inter-country comparison are summarized in Table~\ref{tab:Summary of findings}.

\begingroup
\begin{spacing}{.9}
\renewcommand
\arraystretch{1.5}
\begin{longtable}[ht!]{p{0.16\textwidth} p{0.38\textwidth} p{0.38\textwidth}}
	\caption{Summary of findings}\label{tab:Summary of findings}\\
    \textbf{\textit{Level}}& \textbf{Germany} & \textbf{United Kingdom} \\
    \hline
    \textit{General} & \multicolumn{2}{p{0.79\textwidth}}{
    \begin{minipage}[t]{\linewidth}
    \begin{itemize}[nosep, wide=0pt, leftmargin=*, after=\strut]
    \item Importance of agile methods for government institutions increasing over time and particularly in recent years (295\% increase in number of agile related sites published from 2017 to 2019).
    \item Most agil* sites published (84\%) also mention digital transformation related keywords.
    \item  Both observations are in line with theses in agile related public management literature (cf. e.g. \cite{Mergel}).
    \end{itemize}
    \end{minipage}}\\
    \textit{Intra-Country}& 
    \begin{minipage}[t]{\linewidth}
    \begin{itemize}[nosep, wide=0pt, leftmargin=*, after=\strut]
    \item Agile is rather novel trend; first agil* sites published in 2015; 69 sites in total.
    \item Most active publisher is Federal Ministry of Labour \& Social Affairs (12 pp.) and the state of Baden-Wuerttemberg (10 pp.).
    \item Federal Government Office (9 terms/site) and Bavarian government (5 terms/site) leading in terms of average number of co-search terms per site.
    \end{itemize}
    \end{minipage} &
    \begin{minipage}[t]{\linewidth}
    \begin{itemize}[nosep, wide=0pt, leftmargin=*, after=\strut]
    \item Government institutions have been discussing agile methods on web presences since 2011; 388 sites in total. 
    \item Leading publishers are Government Digital Service (146 pp.) and Cabinet Office (55 pp.).
    \item Highest average number of co-search terms by Home Office (7.34 terms/site) and Education \& Skills Funding Agency (7 terms/site).  
    \end{itemize}
    \end{minipage}\\
    \textit{Inter-Country \newline (only federal ministries)} &
    \multicolumn{2}{p{0.79\textwidth}}{
    \begin{minipage}[t]{\linewidth}
    \begin{itemize}[nosep, wide=0pt, leftmargin=*, after=\strut]
    \item British ministries started publishing agil* sites 4 years earlier; published 4.5 times more sites than German ministries.
    \item Number of British ministries publishing agile related content is twice as high as for Germany; they publish 1.43 times more sites on average; and have a 1.55 times higher total average number of co-search terms per site. 
    \item Most active British ministry (Cabinet office) has 4.6 times more published agil* sites than most active German ministry (Ministry of Labour \& Social Affairs). 
    \item However, German ministry with most in-depth discussion of topic (Federal Government Office) has 1.2 times more co-search terms on average per site than respective British ministry (Home Office).
    \end{itemize}
    \end{minipage}} \\
    &
    \multicolumn{2}{p{0.79\textwidth}}{
    \begin{minipage}[t]{\linewidth}
    \begin{itemize}[nosep, wide=0pt, leftmargin=*, after=\strut]
    \item While British ministries show great knowledge about the particularities of agile methods (many co-search terms related users and iterations), the respective discussion on German ministries' web pages seems rather general and shallow (many co-search terms related to agile ways of working or innovation culture).
    \end{itemize}
    \end{minipage}}\\
    \hline
\end{longtable}
\end{spacing}
\endgroup
\vspace{-0.1cm}


Reconsidering the 








\subsection{Policy Implications}
\subsection{Future Work}
%The limitations of the paper come with suggestions for future research. 
% brittain has a lot of content on blogs --> good outlet to discuss new developments?

% clear strategy (digital strategy) helps?

% no webarchives in germany --> hence, probably lesser hits

% brittain digitalisation from the outside in --> digital as a tool to more efficient and effectively reach fulfill citizens needs (refer back to first ICT strategy article) 

% from mehmet
%findings summary one short paragraph
%contributions and implications for research one paragraph
%implications / recommendation for practice two paragraphs
%limitations of the study one or two paragraphs

%limitations: 
%   no pdfs included in the search algorithm
%   checking each cleaning step and the whole error of all crawls (due to time limits not possible this time) 
% The seed URL for Britain was set to gov.uk, which basically includes the web presences of all British government institutions and agencies. However, for Germany, which does not have a single government domain, only the ministerial websites have been set as seed URLs, thus leading to an exclusion of potential subordinate authorities like the IT provider for the federal government, \href{https://www.itzbund.de/DE/Home/home_node.html}{ITZBund}. Expand the crawl also to subordinate authorities in germany


% blogs as good ways to discuss and try out innovative forms of governance

% research contributions
% shows the evlution and spread of agile methods in relevant parts of the british and German public sector
% new way of doing social science research (venturing into new ways of...)

% potentially do a writers analysis to find out who has been writing those articles (is it only a hand full of people maybe?)

% blog as low key information channel
% werden noch stiefmütterlich behandelt

%nur zwei blogs in deutschland
%https://mb.sachsen-anhalt.de/start/blog-detailansicht/?cHash=76b9e1e8417e5e36ae5105ad14afa4f2&tx_t3extblog_blogsystem%5Baction%5D=show&tx_t3extblog_blogsystem%5Bday%5D=23&tx_t3extblog_blogsystem%5Bmonth%5D=06&tx_t3extblog_blogsystem%5Bpost%5D=267&tx_t3extblog_blogsystem%5Byear%5D=2018


%https://frauenseiten.bremen.de/blog/internationales-sommerstudium-informatica-feminale/

% maybe brittain is more top down thus change from above while germany is more bottom up (change from below) --> need to check in future research

% potentially german government doesn't yet see digital as an integral part to delivering their services to the citizens (hence, digital related approaches are also less valued) 