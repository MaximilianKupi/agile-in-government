\section*{Appendix C}\label{Appendix C}

The general work flow of the Scrapy architecture is as follows (see Figure~\ref{fig:Scrapy Architecture}; adapted from ”Architecture overview” in \cite{Scrapy2018}):
\begin{compactenum}
        \item The Engine gets the initial Requests to crawl from the Spider.
        \item The Engine schedules the Requests in the Scheduler and asks for the next Requests to crawl.
        \item The Scheduler returns the next Requests to the Engine.
        \item The Engine sends the Requests to the Downloader, passing through the Downloader Middlewares.
        \item Once the page finishes downloading the Downloader generates a Response (with that page) and sends it to the Engine, passing through the Downloader Middlewares.
        \item The Engine receives the Response from the Downloader and sends it to the Spider for processing, passing through the Spider Middleware.
        \item The Spider processes the Response and returns scraped items and new Requests (to follow) to the Engine, passing through the Spider Middleware.
        \item The Engine sends processed items to Item Pipelines, then sends processed Requests to the Scheduler and asks for possible next Requests to crawl.
        \item The process repeats (from step 1) until there are no more requests from the Scheduler.
\end{compactenum}\par 
%
\begin{figure}[ht!]
	\centering
	\includegraphics[height=0.4\textwidth]{{"Latex/THESIS/Figures/scrapy_architecture"}.png}
	\caption[Scrapy Architecture]{Scrapy Architecture (Source: \url{https://docs.scrapy.org}, reprinted with permission)}
	\label{fig:Scrapy Architecture}
\end{figure}



\begingroup
\renewcommand
\arraystretch{1.2}
\begin{table}[!htbp]
\centering
\caption{Number of sites per state / country before and after preprocessing}\label{tab:Number of Sites}
\begin{tabular}{>{\raggedleft\arraybackslash}p{5cm} >{\centering\arraybackslash}p{4cm} >{\centering\arraybackslash}p{4cm}}
\multicolumn{1}{l}{\textbf{}} & \textbf{Number of Crawler Downloaded Sites} & \textbf{Number of Sites after Preprocessing / Cleaning} \\ \hline
\textbf{United Kingdom}  & 6185   & 388 \\
\textbf{Germany (total)} & 165384 & 69  \\
Germany Federal                  & 200             & 24           \\
Baden-Wuerttemberg       & 130             & 10           \\
Bavaria                  & 230             & 5            \\
Berlin                   & 70543           & 7            \\
Brandenburg              & 169             & 1            \\
Bremen                   & 1329            & 5            \\
Hamburg                  & 37              & 2            \\
Hesse                    & 64              & 3            \\
Lower Saxony             & 0               & 0            \\
Mecklenburg-Vorpommern   & 4               & 1            \\
North Rhine-Westphalia   & 113             & 0            \\
Rhineland-Palatinate     & 167             & 5            \\
Saarland                 & 59290           & 0            \\
Saxony                   & 33044           & 0            \\
Saxony-Anhalt            & 21              & 3            \\
Schleswig-Holstein       & 31              & 1            \\
Thuringia                & 12              & 2           
\end{tabular}%
\end{table}
\endgroup
