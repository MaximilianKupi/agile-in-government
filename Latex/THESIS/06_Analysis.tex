\section{Analysis}
%
\begin{wrapfigure}{r}{0.5\textwidth}
    \centering
	 \includegraphics[width=0.48\textwidth]{{"Analysis/3_Data_Analysis/visualisations/digital_percentage"}.pdf}
	 \setlength{\belowcaptionskip}{-10pt}	\caption[Percentage of agil* website publications referring to digitalisation in Germany and the UK]{Percentage of agil* website publications referring to digitalisation in Germany and the UK}
	 \label{fig:digital_percentage}
\end{wrapfigure}
% 
\paragraph{General:} On a general, cross-national level, it is to be noted, that the vast majority of crawler yielded sites mentioning agile working methods, also mention digitalisation / digital transformation (see Figure~\ref{fig:digital_percentage}). This is in line with the apparent connection between agile methods and the digital transformation of the public sector, as already discussed in Section~\ref{Digital Transformation}. 

Furthermore, as Figure~\ref{fig:levels and organisation} depicts, the number of crawled sites mentioning agile methods published on German and British government websites has seen a an increasing trend over the years, with a particular steep growth since 2017. Also, the number of different   \begin{wrapfigure}{l}{0.5\textwidth}
    \vspace{-11pt}
    \centering
	 \includegraphics[width=0.48\textwidth]{{"Analysis/3_Data_Analysis/visualisations/number_of_sites_and_organisations_over_time"}.pdf}
	 \setlength{\belowcaptionskip}{-18pt}
	 \caption[Agil* sites published and number of publishing domains over time in Germany and the UK]{Agil* sites published and number of publishing domains over time  in Germany and the UK}
	 \label{fig:levels and organisation}
\end{wrapfigure}
government \mbox{(sub-)}domains publishing agil* sites has been increasing over time. Both these developments can potentially be interpreted as hinting at the the rising relevance of these methods for an increasing number of government organisations in Germany and Britain, and thus can be seen as further confirmation of the themes discussed in current public management literature (see Section~\ref{Digital Transformation}). 

\paragraph{Germany:} As becomes apparent in Figure~\ref{fig:German levels over time} (see next page), German government ministries on the federal and state level, have only started publishing agile related content in 2015. Furthermore, it can be seen, that federal level ministries, by far, have been the most active in that respect (24 pages in total), and the first ones to publish agile related content in 2015.\footnote{For a distinction between all German ministries over time see Figure~\ref{fig:German ministries over time} in \href{Appendix B}{Appendix B}}. The \href{https://www.bundesregierung.de/breg-de/themen/forschung/agil-arbeiten-bei-der-software-entwicklung-276792}{first site published} is an article of the federal government (\href{https://www.bundesregierung.de}{bundesregierung.de}) introducing agile as a new working method in software development and a topic that research relating to Germany's \href{https://www.hightech-strategie.de/de/hightech-strategie-2025-1726.html}{"High Tech Strategy 2025"} needs to pay close attention to as part of the future field innovations of the working world.\footnote{Link: \href{https://www.bundesregierung.de/breg-de/themen/forschung/agil-arbeiten-bei-der-software-entwicklung-276792}{https://www.bundesregierung.de/breg-de/themen/forschung/agil-arbeiten-bei-der-software-entwicklung-276792}} Analysing the crawler yielded sites in Germany makes apparent that the majority of them discuss or mention agile in a similar manner – as something that is (still) seen as mostly "external" to public administration and predominantly relevant for the private sector only.

\begin{figure}[ht!]
	\centering
	 \includegraphics[width=1\textwidth]{{"Analysis/3_Data_Analysis/visualisations/german_levels_over_time"}.pdf}
	 \setlength{\belowcaptionskip}{-10pt}
	 \caption[Agil* sites published by German state and federal ministries over time]{Agil* sites published by German state and federal ministries over time}
	 \label{fig:German levels over time}	
\end{figure}

When comparing the content published by the ministerial web domains of the federal states (see Figure~\ref{fig:map} next page), it can be seen that Baden-Wuerttemberg's ministries have published the most number of websites referring to agile (10 pages), followed by Berlin (7), Bavaria, Bremen, and Rhineland-Palatinate (all 5). However, in terms of how many co-search terms (e.g.'scrum', 'sprint', 'design thinking'; see Table~\ref{tab:Co search terms} in \href{Appendix A}{Appendix A}) appear on average on those sites, Bavaria leads the board with an average of 4 co-search terms per site, followed by Hesse (3.67), and Brandenburg (3). The number of co-search terms might be read as a rough proxy of how in-depth the agile methodology is being described on the respective site. In Bavaria's case the high average number of co-search terms is due to two articles of "Elite Network of Bavaria" – an initiative of the Bavarian State Ministry of Science and the Arts – which describe the methodology in length as part of their educational offers.\footnote{Links: \href{https://www.elitenetzwerk.bayern.de/elitestudiengaenge/aktuelles-esg/artikel0/scrum-workshop-bei-hubert-burda-media-433/?L=248}{https://www.elitenetzwerk.bayern.de/elitestudiengaenge/aktuelles-esg/artikel0/scrum-workshop-bei-hubert-burda-media-433/?L=248}, and \href{https://www.elitenetzwerk.bayern.de/elitestudiengaenge/aktuelles-esg/artikel0/agile-transformation-412/}{https://www.elitenetzwerk.bayern.de/ elitestudiengaenge/aktuelles-esg/artikel0/agile-transformation-412/}.} Again, these articles are not related to agile methods in public administration per se, but discuss the methods as something external. Hesse's high score, however, is due to two pages talking about the methodology in the context of introducing an innovation laboratory for staff of the regional council Kassel.\footnote{Links: \href{https://rp-kassel.hessen.de/forumzukunft}{  https://rp-kassel.hessen.de/forumzukunft}, and \href{https://rp-kassel.hessen.de/\%C3\%BCber-uns/stabsstelle-forum-zukunft/inlab/innovationslabor}{https://rp-kassel.hessen.de/\%C3\%BCber-uns/stabsstelle-forum-zukunft/inlab/innovationslabor}.} Hence, a direct link between the methods and the public administration is established in this case. The scraper did not yield any matching pages for Saarland, North Rhine-Westphalia, Saxony, and Lower Saxony.
\begin{figure}[ht!]
	\centering
	\begin{tabular}{c c}
    Number of Agil* Sites Published & \makecell{Average Number of Co-Search \\ Terms per Site} \\
	\includegraphics[width=0.4\textwidth]{{"Analysis/3_Data_Analysis/visualisations/map_breadth_export"}.pdf} &
	\includegraphics[width=0.4\textwidth]{{"Analysis/3_Data_Analysis/visualisations/map_depth_export"}.pdf}
	\end{tabular}
	\setlength{\belowcaptionskip}{-10pt}
	\caption[Breadth and depth of German federal states' website publications referring to agil*]{Breadth and depth of German federal states' website publications referring to agil*}
	\label{fig:map}
\end{figure}

\paragraph{Britain:} Looking at the matching sites that the scraper yielded from 'gov.uk' reveals the the following patterns over time (see Figure~\ref{fig:British organisations over time} on next page). As in the case of Germany, the very first content on agile methods was published by a central government ministerial department,\footnote{For a distinction between all British central government ministerial departments over time see Figure~\ref{fig:British ministries over time} in \href{Appendix B}{Appendix B}. Grouping of institutions according to \url{https://www.gov.uk/government/organisations}. } namely the Cabinet Office (comparable to the federal government / Bundesregierung) – however, 4 years earlier.\footnote{Link: \url{https://webarchive.nationalarchives.gov.uk/20110410092517/http://innovate.direct.gov.uk/
}.} It is \href{https://webarchive.nationalarchives.gov.uk/20110410092517/http://innovate.direct.gov.uk/
}{the start page of innovate.direct.gov.uk}, a former innovation platform of direct.gov.uk – UK's official website back then – to share "ideas for digital development [...], propose ideas for new projects and keep up to date with our [the development team's] latest work".\footnote{To a certain extent, this development team could be viewed as a predecessor to Government Digital Service.} Interestingly, the second publication was about how Britain's \href{https://webarchive.nationalarchives.gov.uk/20110406081839/http://www.cabinetoffice.gov.uk/content/government-ict-strategy}{new ICT strategy} with a focus on adopting agile methods will help "cutting public spending on Government information and communication technology (ICT) by millions of pounds" – shedding light on a potential motivation for the British government to go agile.\footnote{Link: \url{https://webarchive.nationalarchives.gov.uk/20110406074635/http://www.cabinetoffice.gov.uk/news/government-save-millions-ict}.} In line with that, other publications at the beginning of 2011 are the \href{https://webarchive.nationalarchives.gov.uk/20110406081839/http://www.cabinetoffice.gov.uk/content/government-ict-strategy}{ICT Strategy} itself as well as announcements of \href{https://webarchive.nationalarchives.gov.uk/20110803134807/http://www.cabinetoffice.gov.uk/news/joe-harley-cbe-appointed-chief-information-officer-uk-government}{a change of personnel} and \href{https://webarchive.nationalarchives.gov.uk/20110406074527/http://digitalengagement.cabinetoffice.gov.uk/blog/2011/03/29/towards-a-single-government-domain/}{the plan to develop "a single government domain" (gov.uk)}.\footnote{Links: \url{https://webarchive.nationalarchives.gov.uk/20110406081839/http://www.cabinetoffice.gov.uk/content/government-ict-strategy}, \url{https://webarchive.nationalarchives.gov.uk/20110803134807/http://www.cabinetoffice.gov.uk/news/joe-harley-cbe-appointed-chief-information-officer-uk-government}, and \url{https://webarchive.nationalarchives.gov.uk/20110406074527/http://digitalengagement.cabinetoffice.gov.uk/blog/2011/03/29/towards-a-single-government-domain/}.} Besides the Cabinet Office, the only other institution publishing agile methods related content in 2011 was Government Digital Service. This "centre of excellence" was founded in 2011 by the Cabinet Office and commissioned to take the leading role in the British government's digital transformation (\cite{GovernmentDigitalService2020}). Consequently, their first crawler yielded publications have been about \href{https://webarchive.nationalarchives.gov.uk/20120404152514/http://digital.cabinetoffice.gov.uk/2011/10/25/the-second-lever/}{them wanting to hire "world-class digital talent"}, \href{https://webarchive.nationalarchives.gov.uk/20120404161123/http://digital.cabinetoffice.gov.uk/2011/12/08/new-home-for-gds/}{moving into their new office} or \href{https://webarchive.nationalarchives.gov.uk/20121102172608/http://digital.cabinetoffice.gov.uk/2011/09/19/introducing-the-needotron-working-out-the-shape-of-the-product/}{introducing specific tools applied for agile development}.\footnote{Links: \url{https://webarchive.nationalarchives.gov.uk/20120404161123/http://digital.cabinetoffice.gov.uk/2011/12/08/new-home-for-gds/}, \url{https://webarchive.nationalarchives.gov.uk/20120404152514/http://digital.cabinetoffice.gov.uk/2011/10/25/the-second-lever/}, and \url{https://webarchive.nationalarchives.gov.uk/20121102172608/http://digital.cabinetoffice.gov.uk/2011/09/19/introducing-the-needotron-working-out-the-shape-of-the-product/}} 
\begin{figure}[ht!]
	\centering
	 \includegraphics[width=1\textwidth]{{"Analysis/3_Data_Analysis/visualisations/british_organisations_over_time"}.pdf}
	 \setlength{\belowcaptionskip}{-10pt}
	 \caption[Agile* websites published over time on gov.uk, distinguished between organisations]{Agile* websites published over time on gov.uk, distinguished between organisations}
	 \label{fig:British organisations over time}
\end{figure}
%
It is to be noted that both, the Cabinet Office and the Government Digital Service, are the only institutions on gov.uk consistently publishing agile related content every year since 2011 (a total of 49 and 118 sites respectively). The only domain publishing more content on agile than Government Digital Service was the general domain, gov.uk.\footnote{The British government uses the general domain (without any sub-domain specification) to publish general guidance, speeches or information on the topic.} 

In terms of average number of co-search terms, the Home Office clearly leads the ranking with 9 co-search terms per site, followed by the Department for Work \& Pensions and the the Education \& Skills Funding Agency (both 7) (see Figure~\ref{fig:British agile depth organisation} in \href{Appendix B}{Appendix B}). However, this result is slightly misleading as the crawler only yielded one page for the Home Office which, in addition, is \href{https://hodigital.blog.gov.uk/category/agile/}{an overview page about agile}.\footnote{Link: \url{https://hodigital.blog.gov.uk/category/agile/}.}

Over time, the crawler yielded sites show a similar pattern than in the case of Germany: The number of organisations publishing content on agile increases over time, in particular after 2016. However, non-central government agencies in Britain are (still) significantly less active in that respect – the councils of Durham, Essex, Leicester and North Yorkshire as well as the Greater London Authority and the Local Government Authority published only one site each. Also, it seems like the election cycle might be somewhat correlated with the number of published sites: One year after the general elections (in 2015 and 2017) there is a notable increase in sites published.\footnote{The first notable increase in 2013 could potentially also be linked to the general elections in 2010, however the effects are only noticeable with a delay, due to Government Digital Service needing to ramp up their work. Congruently to the central government's behaviour, local authorities also only started publishing agile related content after the local elections in 2018. Additionally, the spread of the methodology to the regional government might be explained by government experts from the field moving to the countryside in their mid-thirties/forties to have better "family life conditions", after gaining first work experience at GDS.} Lastly, it is remarkable that 41 percent of the agile content was published on the online blogs of the respective organisations.
\paragraph{Britain vs Germany:} The seed URL for Britain was set to gov.uk, which basically includes the web presences of all British government institutions. However, for Germany, only the the ministerial websites on the federal and state level have been set as as seed URLs. Thus, in order to have 

%
\begin{wrapfigure}{r}{0.5\textwidth}
	\centering
	 \includegraphics[width=0.45\textwidth]{{"Analysis/3_Data_Analysis/visualisations/british_vs_German_federal_ministries_over_time"}.pdf}
	 \caption[Agile* websites published by British and German ministries over time]{Agile* websites published by British and German ministries over time}
	 \label{fig:Agile* websites published over time by British and German ministries over time}
\end{wrapfigure}
%

% figure wird außen abgeschnitten (text der achsen)
\begin{figure}[ht!]
	\centering
	 \includegraphics[width=1\textwidth]{{"Analysis/3_Data_Analysis/visualisations/Comparison_Federal_Ministries_without_names"}.pdf}
	 \caption[Breadth and depth of agile* websites published by British and German ministries]{Breadth and depth of agile* websites published by British and German ministries}
	 \label{fig:Breadth and depth of agile* websites published by British and German ministries}
\end{figure}
%
%
\begin{figure}[ht!]
	\centering
    \begin{tabular}{c c}
    Germany & UK\\
    \includegraphics[width=0.45\textwidth]{{"Analysis/3_Data_Analysis/visualisations/wordcloud_germany_ministries"}.pdf} & \includegraphics[width=0.45\textwidth]{{Analysis/3_Data_Analysis/visualisations/wordcloud_uk_ministries"}.pdf}
    \end{tabular}
%	\textbf{Test}\par\medskip
%	\includegraphics[height=0.35\textwidth]{{"Analysis/3_Data_Analysis/wordcloud_germany"}.pdf}
%	\includegraphics[height=0.35\textwidth]{{"Analysis/3_Data_Analysis/wordcloud_uk"}.pdf}
	\caption[Frequency of co-search terms on agile* websites published by British and German ministries]{Frequency of co-search terms on agile* websites published by British and German ministries}
	\label{fig:wordclouds}
\end{figure}
%

