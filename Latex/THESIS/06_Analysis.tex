\section{Analysis}
First and foremost, it is to be noted, that the vast majority of sites mentioning agile working methods, also mention digitalisation / digital transformation (see Figure~\ref{fig:digital_percentage}). This is in line with the apparent connection between agile methods and the digital transformation of the public sector, as already discussed in the Section~\ref{Digital Transformation}.\par
%
\begin{wrapfigure}{r}{0.5\textwidth}
    \centering
	 \includegraphics[width=0.48\textwidth]{{"Analysis/3_Data_Analysis/visualisations/digital_percentage"}.pdf}
	 \caption[Percentages of agil* website publications referring to digitalisation]{Percentages of agil* website publications referring to digitalisation}
	 \label{fig:digital_percentage}
\end{wrapfigure}
%
When looking at the crawling results from Germany, it 
\begin{wrapfigure}{l}{0.5\textwidth}
    \centering
	 \includegraphics[width=0.48\textwidth]{{"/Analysis/3_Data_Analysis/visualisations/german_levels_over_time"}.pdf}
	 \caption[Agile* websites published over time, distinguished between state and federal government]{Agile* websites published over time, distinguished between states and federal}
	 \label{fig:digital_percentage}
\end{wrapfigure}
