\section{Analysis}

% shrift in figure noch größer machen
%
\begin{wrapfigure}{r}{0.5\textwidth}
    \centering
	 \includegraphics[width=0.48\textwidth]{{"Analysis/3_Data_Analysis/visualisations/digital_percentage"}.pdf}
	 \setlength{\belowcaptionskip}{-10pt}	\caption[Percentage of agil* website publications referring to digitalisation in Germany and the UK]{Percentage of agil* website publications referring to digitalisation in Germany and the UK}
	 \label{fig:digital_percentage}
\end{wrapfigure}
% 
\paragraph{General:} On a general, cross-national level, it is to be noted, that the vast majority of crawler yielded sites mentioning agile working methods, also mention digitalisation / digital transformation (see Figure~\ref{fig:digital_percentage}). This is in line with the apparent connection between agile methods and the digital transformation of the public sector, as already discussed in Section~\ref{Digital Transformation}. 

Furthermore, as Figure~\ref{fig:levels and organisation} depicts, the number of crawled sites mentioning agile methods published on German and British government websites has seen a an increasing trend over the years, with a particular steep growth since 2017. Also, the number of different   \begin{wrapfigure}{l}{0.5\textwidth}
    \vspace{-11pt}
    \centering
	 \includegraphics[width=0.48\textwidth]{{"Analysis/3_Data_Analysis/visualisations/number_of_sites_and_organisations_over_time"}.pdf}
	 \setlength{\belowcaptionskip}{-18pt}
	 \caption[Agil* sites published and number of publishing domains over time in Germany and the UK]{Agil* sites published and number of publishing domains over time  in Germany and the UK}
	 \label{fig:levels and organisation}
\end{wrapfigure}
government \mbox{(sub-)}domains publishing agil* sites has been increasing over time. Both these developments can potentially be interpreted as hinting at the the rising relevance of these methods for an increasing number of government organisations in Germany and Britain, and thus can be seen as further confirmation of the themes discussed in current public management literature (see Section~\ref{Digital Transformation}). 

\paragraph{Germany:} As becomes apparent in Figure~\ref{fig:German levels over time} (see next page), German government ministries on the federal and state level, have only started publishing agile related content in 2015. Furthermore, it can be seen, that federal level ministries, by far, have been the most active in that respect (24 pages in total), and the first ones to publish agile related content in 2015. 

\begin{figure}[ht!]
	\centering
	 \includegraphics[width=1\textwidth]{{"Analysis/3_Data_Analysis/visualisations/german_levels_over_time"}.pdf}
	 \caption[Agil* sites published by German state and federal ministries over time]{Agil* sites published by German state and federal ministries over time}
	 \label{fig:German levels over time}	
\end{figure}

When comparing the content published by the ministerial web domains of the federal states, it can be seen that Baden-Wuerttemberg's ministries have published the most number of websites referring to agile (10 pages), followed by Berlin (7), Bavaria (6), and Bremen (5). However, in terms of how many co-search terms (for example 'scrum', 'sprint', 'design thinking'; see Table~\ref{tab:Co search terms} in \href{Appendix A}{Appendix A}) appear on average on those sites, Hesse leads the board with 3.67 co-search terms, followed by Bavaria (3.5), and Brandenburg (3). The number of co-search terms might be read as a rough proxy of how in-depth the agile methodology is being described on the respective website. The scraper did not yield any matching pages for Saarland, North Rhine-Westphalia, Saxony, and Lower Saxony.
\begin{figure}[ht!]
	\centering
	\begin{tabular}{c c}
    Number of Agil* Sites Published & \makecell{Average Number of Co-Occurring \\ Terms per Site} \\
	\includegraphics[width=0.4\textwidth]{{"Analysis/3_Data_Analysis/visualisations/map_breadth_export"}.pdf} &
	\includegraphics[width=0.4\textwidth]{{"Analysis/3_Data_Analysis/visualisations/map_depth_export"}.pdf}
	\end{tabular}
	\setlength{\belowcaptionskip}{-18pt}
	\caption[Breadth and depth of German federal states' website publications referring to agil*]{Breadth and depth of German federal states' website publications referring to agil*}
	\label{fig:map}
\end{figure}

\paragraph{Britain:} Looking at the matching hits that the scraper yielded from 'gov.uk' reveals the the following patterns over time (see Figure~\ref{fig:British organisations over time} on next page): The very first web-content on the topic, was a \href{https://www.gov.uk/government/publications/the-way-we-work-tw3-best-practice-guidelines-for-smarter-working}{guidance on smarter working} published on the general domain in 2004.\footnote{As this page was updated since its first publication, it might well be, that the original content was not mentioning agile. However, this cannot be inferred from the available web information alone, but would require further investigation} The general domain is also the source for most (WORK IN PROGRESS)

% most hits
% regional hits (few)
% should i print the 
%
\begin{figure}[ht!]
	\centering
	 \includegraphics[width=1\textwidth]{{"Analysis/3_Data_Analysis/visualisations/british_organisations_over_time"}.pdf}
	 \caption[Agile* websites published over time on gov.uk, distinguished between organisations]{Agile* websites published over time on gov.uk, distinguished between organisations}
	 \label{fig:British organisations over time}
\end{figure}
%
% maybe use wrapfigure and put this one to the right
\begin{figure}[ht!]
	\centering
	 \includegraphics[width=0.45\textwidth]{{"Analysis/3_Data_Analysis/visualisations/british_vs_German_federal_ministries_over_time"}.pdf}
	 \caption[Agile* websites published by British and German ministries over time]{Agile* websites published by British and German ministries over time}
	 \label{fig:Agile* websites published over time by British and German ministries over time}
\end{figure}
%

% figure wird außen abgeschnitten (text der achsen)
\begin{figure}[ht!]
	\centering
	 \includegraphics[width=1\textwidth]{{"Analysis/3_Data_Analysis/visualisations/Comparison_Federal_Ministries_without_names"}.pdf}
	 \caption[Breadth and depth of agile* websites published by British and German ministries]{Breadth and depth of agile* websites published by British and German ministries}
	 \label{fig:Breadth and depth of agile* websites published by British and German ministries}
\end{figure}
%
%
\begin{figure}[ht!]
	\centering
    \begin{tabular}{c c}
    Germany & UK\\
    \includegraphics[width=0.45\textwidth]{{"Analysis/3_Data_Analysis/visualisations/wordcloud_germany_ministries"}.pdf} & \includegraphics[width=0.45\textwidth]{{Analysis/3_Data_Analysis/visualisations/wordcloud_uk_ministries"}.pdf}
    \end{tabular}
%	\textbf{Test}\par\medskip
%	\includegraphics[height=0.35\textwidth]{{"Analysis/3_Data_Analysis/wordcloud_germany"}.pdf}
%	\includegraphics[height=0.35\textwidth]{{"Analysis/3_Data_Analysis/wordcloud_uk"}.pdf}
	\caption[Frequency of co-search terms on agile* websites published by British and German ministries]{Frequency of co-search terms on agile* websites published by British and German ministries}
	\label{fig:wordclouds}
\end{figure}
%

