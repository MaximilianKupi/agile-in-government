\section{Analysis}

%
\begin{wrapfigure}{r}{0.5\textwidth}
    \centering
	 \includegraphics[width=0.48\textwidth]{{"Analysis/3_Data_Analysis/visualisations/digital_percentage"}.pdf}
	 \caption[Percentages of agil* website publications referring to digitalisation in Germany and the UK]{Percentages of agil* website publications referring to digitalisation in Germany and the UK}
	 \label{fig:digital_percentage}
\end{wrapfigure}
% 
First and foremost, it is to be noted, that the vast majority of sites mentioning agile working methods, also mention digitalisation / digital transformation (see Figure~\ref{fig:digital_percentage}). This is in line with the apparent connection between agile methods and the digital transformation of the public sector, as already discussed in the Section~\ref{Digital Transformation}.\par
As becomes apparent in Figure~\ref{fig:German levels over time}, German governments on federal and state level, have only started publishing agile related content in 2015. Furthermore, it can be seen, that federal level ministries by far, have been the most active in that respect (24 pages in total), and the first ones to publish agile related content in 2015.\par
%
\begin{figure}[ht!]
	\centering
	 \includegraphics[width=1\textwidth]{{"Analysis/3_Data_Analysis/visualisations/german_levels_over_time"}.pdf}
	 \caption[Agile* websites published in Germany over time, distinguished between state and federal government]{Agile* websites published in Germany over time, distinguished between states and federal}
	 \label{fig:German levels over time}
\end{figure}
%
When comparing the content published by the ministerial web domains of the federal states, it can be seen that Baden-Wuerttemberg's ministries have published the most number of websites referring to agile (12 pages), followed by Bavaria (8), Bremen and Berlin (each 7). However, in terms of how many co-search terms (for example 'scrum', 'sprint', 'design thinking'; see Table~\ref{tab:Co search terms} in \href{Appendix A}{Appendix A}) appear on average on those sites, Hesse leads the board with 3.25 co-search terms, followed by Bavaria (2.27), and Rhineland-Palatine (2.7). The number of co-search terms might be read as a rough proxy of how in-depth the agile methodology is being described on the respective website. The scraper didn't yield any matching pages for Saarland, North Rhine-Westphalia, Saxony, and Lower Saxony.\par   
\begin{figure}[ht!]
	\centering
	\begin{tabular}{c c}
    Number of Agil* Websites Published & \makecell{Average Number of Co-Occurring \\ Terms per Website} \\
	\includegraphics[width=0.4\textwidth]{{"Analysis/3_Data_Analysis/visualisations/map_breadth_export"}.pdf} &
	\includegraphics[width=0.4\textwidth]{{"Analysis/3_Data_Analysis/visualisations/map_depth_export"}.pdf}
	\end{tabular}
	\caption[Breadth and depth of German federal states' website publications referring to agil*]{Breadth and depth of German federal states' website publications referring to agil*}
	\label{fig:map}
\end{figure}
%
Looking at the matching hits that the scraper yielded from 'gov.uk' reveals the the following patterns over time (see Figure~\ref{fig:British organisations over time}): The very first web-content on the topic, was a \hyperlink{https://www.gov.uk/government/publications/the-way-we-work-tw3-best-practice-guidelines-for-smarter-working}{guidance on smarter working} published on the general domain in 2004.\footnote{Since this page was updated since its first publication, it might well be, that the original content was not mentioning agile.} The general domain is also the source for most 
%
\begin{figure}[ht!]
	\centering
	 \includegraphics[width=1\textwidth]{{"Analysis/3_Data_Analysis/visualisations/british_organisations_over_time"}.pdf}
	 \caption[Agile* websites published over time on gov.uk, distinguished between  organisations]{Agile* websites published over time on gov.uk, distinguished between  organisations}
	 \label{fig:British organisations over time}
\end{figure}
%